\subsection*{Função chão interna}

Devido às propriedades da função chão,
podemos reescrever $a_{k+1}$ como:
\[
  a_{k+1}
  = \bracketize[.75em]{\lfloor}{\rfloor}{
      \dfrac{\left\lfloor
               (n-1) a_k + \dfrac{x}{a_k^{n-1}}
             \right\rfloor}
            {n}
    } \\
  = \bracketize[.75em]{\lfloor}{\rfloor}{
      \dfrac{(n-1) a_k +
             \left\lfloor \dfrac{x}{a_k^{n-1}} \right\rfloor}
            {n}
    } \\
\]
Pois o denominador $n$ é um inteiro positivo,
e a parcela $(n - 1) a_k$ é inteira.
Essa nova formulação corresponde precisamente
ao algoritmo inicialmente proposto:
\[
  a_{k+1} = \left\lfloor \dfrac{(n-1) a_k + d_{k+1}}{n} \right\rfloor,
  \quad\text{onde}\quad
  d_{k+1} = \left\lfloor \dfrac{x}{a_k^{n-1}} \right\rfloor \\
\]

Isso significa que a aplicação da função chão interna
na definição da recorrência $a_k$ não altera em nada o falor de $a_k$,
desde que a função chão externa seja mantida.
