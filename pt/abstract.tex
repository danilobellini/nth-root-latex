\section*{Resumo}

Na PEP 572, há um algoritmo\footnote{
  \url{https://www.python.org/dev/peps/pep-0572/\#a-numeric-example}
}
implementado no Python com e sem uma \emph{assignment expression}
para o cálculo da raiz $n$-ésima de $x$,
utilizando apenas inteiros (entrada, saída e valores intermediários),
sendo que o resultado final é o valor truncado da raiz desejada
no caso deste não ser inteiro.
O objetivo deste texto é explicar, justificar, provar e analisar
esse algoritmo e seu funcionamento.
