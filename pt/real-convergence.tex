\subsection*{Convergência em $\mathds{R}$
             sem truncamentos intermediários}

Esse processo iterativo,
obtido a partir do método de Newton-Raphson,
converge?
Para $n = 1$,
aplicando esse valor em \eqref{real-recurrence}
encontramos $\alpha_{k+1} = \delta_{k+1} = x$,
isto é, a sequência converge em um único passo,
e o palpite inicial $\alpha_0$ é completamente ignorado,
de maneira que equacionamento não será realmente uma recorrência,
mas uma mera atribuição para o resultado.
E para $n \ge 2$?

Admitindo que $n \ge 2$, e sabendo que $x > 0$,
suponha que $\alpha_k > 0$.
Nesse caso, podemos notar que $\alpha_{k+1} > 0$,
visto que $n - 1 \ge 1$ e as somas, multiplicações e divisões
que compõe a recorrência \eqref{real-recurrence}
são todas realizadas apenas sobre números positivos,
garantindo um resultado positivo.
Por indução, sabemos que se o palpite inicial $\alpha_0$ for positivo,
a sequência inteira será formada apenas por números positivos.
Além disso, a equação \eqref{real-derivative} é estritamente positiva,
de maneira que $f'(\alpha_k) \ne 0$ para todo $\alpha_k$,
o que é suficiente para dizermos que todos os elementos da sequência
podem ser construídos.

Seja $\mathfrak{B}$ uma amostra
contendo $n - 1$ elementos iguais a $\alpha_k$
e um único elemento $\delta_{k+1}$.
Segundo a desigualdade entre as médias aritmética e geométrica,
os $n$ elementos dessa amostra,
$\mathfrak{b}_1 = \mathfrak{b}_2 = ... = \mathfrak{b}_{n-1}
                = \alpha_{k+1}$
e $\mathfrak{b}_0 = \delta_k$,
satisfazem
$\sum \mathfrak{b}_j/n \ge \sqrt[n]{\prod \mathfrak{b}_j}$,
ou seja:
\begin{equation}
  \dfrac{(n-1) \alpha_k + \delta_{k+1}}{n} \ge
  \sqrt[n]{\alpha_k^{n-1} \delta_{k+1}}
\end{equation}
Usando as definições de $\delta_{k+1}$ e $\alpha_{k+1}$
presentes em \eqref{real-recurrence},
essa mesma desigualdade pode ser escrita como:
\begin{equation}\label{real-amgm-min-alphak+1}
  \alpha_{k+1} \ge \sqrt[n]{x}
\end{equation}
Esse resultado nos permite reduzir o domínio de análise,
pois $0 < \alpha_0 < \sqrt[n]{x} \implies \alpha_1 \ge \sqrt[n]{x}$,
e nesse caso poderíamos simplesmente tomar $\alpha_1$
como o novo valor inicial, e teríamos essencialmente a mesma sequência,
a menos da defasagem por $1$ nos índices.
Dito de outra forma,
se a sequência convergir para qualquer $\alpha_0 \ge \sqrt[n]{x}$,
então ela irá convergir para qualquer $\alpha_0 > 0$,
e podemos continuar a avaliação
somente com o cenário em que $\alpha_0 \ge \sqrt[n]{x}$.

Quando $\alpha_k = \sqrt[n]{x}$, não há o que analisar,
pois a sequência definida em \eqref{real-recurrence} resulta em
$\delta_{k+1} = \alpha_{k+1} = \sqrt[n]{x}$.
Uma sequência constante é sempre trivialmente convergente
para o valor constante de seus elementos.

Por outro lado, quando $\alpha_k > \sqrt[n]{x}$,
\[
  \alpha_k > \sqrt[n]{x}
  \quad\stackfirst{n \ge 2}{\iff}\quad
  \alpha_k^n > {x}
  \quad\stackfirst{\alpha_k > 0}{\iff}\quad
  \alpha_k > \dfrac{x}{\alpha_k^{n - 1}}
\]
Em que a última expressão à direita é precisamente
como $\delta_{k+1}$ foi definido em \eqref{real-recurrence}.
Essa desigualdade $\alpha_k > \delta_{k+1}$
e a definição de $\alpha_{k+1}$ dada em \eqref{real-recurrence}
permitem escrever:
\[
  \alpha_k > \sqrt[n]{x}
  \quad\implies\quad
  \alpha_{k+1} = \dfrac{(n - 1) \alpha_k + \delta_{k+1}}{n}
  < \dfrac{(n-1) \alpha_k + \alpha_k}{n} = \alpha_k
\]
Resumidamente:
\begin{equation}\label{real-monotone}
  \alpha_k > \sqrt[n]{x}
  \quad\iff\quad
  \alpha_k > \delta_{k+1}
  \quad\implies\quad
  \alpha_{k+1} < \alpha_k
\end{equation}

Juntando os resultados \eqref{real-amgm-min-alphak+1}
e \eqref{real-monotone}, temos:
\begin{equation}
    \alpha_k > \sqrt[n]{x}
  \quad\implies\quad
      \tikzeq{real-conv-a}{\sqrt[n]{x}}
    \le
      \tikzeq{real-conv-b}{\alpha_{k+1}}
    <
      \tikzeq{real-conv-c}{\alpha_k}
  \rule[-1.5em]{0pt}{3.5em}
  \begin{tikzpicture}[eq-overlay]
    \draw[brace]
      ($(real-conv-a.base west) + (0, .7em)$)
      -- node[above=.3em] {Limitada}
      ($(real-conv-b.base east) + (0, .7em)$);
    \draw[brace, decoration={mirror}]
      ($(real-conv-b.base west) + (0, -.15em)$)
      -- node[below=.2em] {Monotônica}
      ($(real-conv-c.base east) + (0, -.15em)$);
  \end{tikzpicture}
\end{equation}
Isto é, a cada passo/iteração estamos mais próximos do resultado.
Como tal sequência é \emph{limitada} e \emph{monotônica},
sabemos que ela é \emph{convergente}.
Tal resultado completa todos os casos,
permitindo-nos dizer que a sequência é convergente
para qualquer valor inicial $\alpha_0 > 0$,
desde que $n, x \in \mathds{N}^*$.

O valor para o qual a sequência converge, o limite da sequência,
como já foi discutido
após a obtenção do equacionamento do método de Newton-Raphson,
é uma raiz da função $f(\alpha)$.
Ao calcular a derivada de $f(\alpha)$ em \eqref{real-derivative},
já foi visto que a raiz da função é única em $\mathds{R}^+$,
visto que a função é contínua e monotônica crescente no domínio,
além de ``cruzar'' de um valor negativo para um valor positivo
nesse domínio.
Isso é suficiente
para dizer que $\alpha_k \to \sqrt[n]{x}$ quando $k \to \infty$,
mas podemos obter o mesmo resultado
substituindo $\alpha_k$ e $\alpha_{k+1}$ por um valor $L$
na recorrência \eqref{real-recurrence},
visto que $(\alpha_{k+1} - \alpha_k) \to 0$ quando a sequência converge,
e com isso obtemos:
\begin{equation}
    L = \dfrac{(n - 1) L + \dfrac{x}{L^{n-1}}}{n}
  \stackfirst{n \ne 0}{\iff}
    nL = nL - L + \dfrac{x}{L^{n-1}}
  \iff
    L = \dfrac{x}{L^{n-1}}
  \stackfirst{L \ne 0}{\iff}
    L^n = x
  \stackfirst{L > 0}{\iff}
    L = \sqrt[n]{x}
\end{equation}
