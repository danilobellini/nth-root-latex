\subsection*{Convergência de sequências}

Uma \emph{sequência} de números reais
é uma função $q:\mathds{N}\to\mathds{R}$
para a qual denotamos como $q_k$ seu $k$-ésimo elemento,
também chamado de \emph{termo geral}.
Podemos construir uma sequência por meio de uma \emph{recorrência},
que consiste em definir cada novo elemento da sequência
como uma função dos elementos anteriores,
desde que fixados valores iniciais
que permitam construir o restante da sequência.
Em particular, o algoritmo proposto
está escrito na forma de uma recorrência.

Uma sequência é dita:
\begin{itemize}
  \item
    \emph{Limitada} se existe $M \ge 0$ real
    tal que $|q_k| \le M$ para todo $k$.
    Ou, equivalentemente, se existem $M_{\min}$ e $M_{\max}$
    tais que  $M_{\min} \le q_k \le M_{\max}$ para todo $k$.
  \item
    \emph{Crescente} ou \emph{estritamente crescente}
    quando $q_{k+1} > q_k$.
  \item
    \emph{Não-crescente} quando $q_{k+1} \le q_k$.
  \item
    \emph{Decrescente} ou \emph{estritamente decrescente}
    quando $q_{k+1} < q_k$.
  \item
    \emph{Não-decrescente} quando $q_{k+1} \ge q_k$.
  \item
    \emph{Monotônica} ou \emph{monótona} se, para todo $k$,
    ela for ou não-crescente, ou não-decrescente.
    Isso também inclui os casos
    em que a sequência é ou estritamente crescente,
    ou estritamente decrescente.
  \item
    \emph{Convergente} quando
    existe um \emph{limite} $L$ para a sequência,
    isto é, quando,
    para qualquer $\varepsilon > 0$,
    existe um $K$ tal que $|q_k - L| < \varepsilon$
    para todo $k \ge K$.
    Tal limite pode ser indicado como
    ``$q_k \to L$ quando $k \to \infty$'',
    ou $\displaystyle \lim_{k\to\infty} q_k = L$.
  \item
    \emph{Divergente} quando não é convergente.
\end{itemize}

\textbf{Se a sequência é convergente, seu limite é único}.
Para provar, suponha que existam dois limites distintos, $L_1$ e $L_2$,
e, sem perda de generalidade, admita $L_2 > L_1$.
Isso se revela uma contradição,
pois adotando qualquer valor de $\varepsilon$
que satisfaça $0 < \varepsilon \le (L_2 - L_1)/2$,
podemos justificar, por contraposição,
que pelo menos um dos limites é falso:
\[
  \begin{array}{l}
        |q_k - L_1| < \varepsilon
      \iff
        - \varepsilon < q_k - L_1 < \varepsilon
      \iff
        L_1 - \varepsilon <
        \overbrace{q_k < L_1 + \varepsilon}^{(\varepsilon_1)}
    \\
        |q_k - L_2| < \varepsilon
      \iff
        - \varepsilon < q_k - L_2 < \varepsilon
      \iff
        \underbrace{L_2 - \varepsilon < q_k}_{(\varepsilon_2)}
        < L_2 + \varepsilon
  \end{array}
  \Bigg\}
  \stackfirst{\mathclap{
    (\varepsilon_1) \text{ e } (\varepsilon_2)
  }}{\implies}
    L_2 - \varepsilon < L_1 + \varepsilon
  \iff
    \varepsilon > \dfrac{L_2 - L_1}{2}
\]

\textbf{Se a sequência é convergente, então ela é limitada}.
Como a sequência é convergente,
há um limite $L$ para o qual ela converge.
Fixe um valor real positivo para o $\varepsilon$.
A definição do limite nos diz que existe um $K$
a partir do qual vale $|q_k - L| < \varepsilon$,
o que nos permite particionar a sequência em duas:
os $K$ elementos iniciais da sequência,
e os elementos a partir dos quais
sempre vale $|q_k - L| < \varepsilon$.
Tome
$\displaystyle
 M = \max\left(|L| + \varepsilon, \max_{0 \le k < K} |q_k|\right)$,
então $|q_k| \le M$ para todo $k$,
pois o conjunto dos $K$ elementos iniciais da sequência é limitado
(i.e., possui valores máximo e mínimo)
e os elementos seguintes podem ser majorados por $|L| + \varepsilon$,
que é um número positivo:
\[
  \begin{array}{c}\displaystyle
      |q_k - L| < \varepsilon
    \iff
      - \varepsilon < q_k - L < \varepsilon
    \iff
      \tikzeq{seqlimit-above}{
        L - \varepsilon < q_k < L + \varepsilon
      }
    \\[1em] \displaystyle
      -|L| - \varepsilon \le
        \tikzeq{seqlimit-below}{
          L - \varepsilon < q_k < L + \varepsilon
        }
      \le |L| + \varepsilon
    \implies
      -(|L| + \varepsilon) < q_k < |L| + \varepsilon
    \iff
      q_k < |L| + \varepsilon
    \begin{tikzpicture}[eq-overlay]
      \draw[brace80]
        (seqlimit-above.south east) -- (seqlimit-above.south west);
      \draw[brace80]
        (seqlimit-below.north west) -- (seqlimit-below.north east);
      \coordinate (seqlimit-below-tip) at
        ($(seqlimit-below.north west)!.8!(seqlimit-below.north east)
          + (0, .5em)$);
      \coordinate (seqlimit-above-tip) at
        ($(seqlimit-above.south east)!.8!(seqlimit-above.south west)
          + (0, -.5em)$);
      \draw[out=210, in=30, distance=2em, ->,
            arrows=-{Latex[length=2mm, width=1mm]}]
        (seqlimit-above-tip) to (seqlimit-below-tip);
    \end{tikzpicture}
  \end{array}
\]

\textbf{Se a sequência é monotônica e limitada,
        então ela é convergente}.
Dizer que a sequência é limitada
corresponde a dizer que o conjunto de todos os elementos da sequência
é limitado tanto superiormente como inferiormente.
A completude dos números reais\footnote{
  O axioma da completude dos números reais
  também é conhecido como axioma do supremo,
  propriedade do menor limitante superior,
  ou cota superior.
  Esse axioma diz que
  todo conjunto de números reais que é limitado superiormente
  possui um número real como menor limitante superior,
  chamado de \emph{supremo},
  mesmo que este não pertença ao próprio conjunto.
  Isto significa dizer que,
  dado um conjunto $B \subset \mathds{R}$ limitado superiormente,
  para todo $\varepsilon > 0$
  existe um $b \in B$
  tal que $b > (\sup B) - \varepsilon$.
  Analogamente
  para conjuntos de números reais limitados inferiormente,
  o \emph{ínfimo} é o maior limitante inferior do conjunto,
  como consequência da aplicação desse axioma
  sobre $C = \{-b: b \in B\}$,
  isto é,
  dado um conjunto $C \subset \mathds{R}$ limitado inferiormente,
  para todo $\varepsilon > 0$
  existe um $c \in C$
  tal que $c < (\inf C) + \varepsilon$.
} garante que, nesse caso, o conjunto de valores reais $q_k$
possui um supremo e um ínfimo.
Caso seja não-decrescente, a sequência converge para o supremo
$S = \sup \{q_k\}$,
e caso seja não-crescente, a sequência converge para o ínfimo
$I = \inf \{q_k\}$,
pois:
\[
  \begin{array}{r|rcl|rcl|c}
      \multicolumn{1}{r}{} &  % Remove border
      \multicolumn{3}{c}{\textbf{Não-decrescente}} &
      \multicolumn{3}{c}{\textbf{Não-crescente}}
    \\ \cline{2-7}
      \text{Completude (axioma)}
      &
      \forall \varepsilon \exists k_s :
        \varepsilon > 0 &\implies& q_{k_s} > S - \varepsilon
      &
      \forall \varepsilon \exists k_i :
        \varepsilon > 0 &\implies& q_{k_i} < I + \varepsilon
      & \text{(A)}
    \\
      \text{Limitada pelo supremo/ínfimo}
      &
      && S \ge q_k
      &
      && I \le q_k
      & \text{(L)}
    \\
      \text{Monotônica}
      &
      k > k_s &\implies& q_k \ge q_{k_s}
      &
      k > k_i &\implies& q_k \le q_{k_i}
      & \text{(M)}
    \\ \cline{2-7}
  \end{array}
\]
\[
  \begin{array}{r@{\qquad}l}
      \hfill\textbf{Monotônica não-decrescente e limitada}\hfill
    &
      \hfill\textbf{Monotônica não-crescente e limitada}\hfill
    \\[1.5em]
      \underbracket{
        \forall \varepsilon \exists k_s :
        \substack{
          (\varepsilon > 0) \\[.3em]
          \text{e} \\[.3em]
          (k > k_s)
        }
        {\displaystyle\Bigg\}}
        {\Rightarrow}
        \left\{
          \begin{array}{c}
              \tikzeq{sup-a}{S - \varepsilon}
                <
              \tikzeq{sup-b}{q_{k_s}}
                \le
              \tikzeq{sup-c}{q_k}
                \le
              \tikzeq{sup-d}{S}
                <
              \tikzeq{sup-e}{S + \varepsilon}
              \begin{tikzpicture}[eq-overlay]
                \draw[brace]
                  ($(sup-a.base west) + (0, .7em)$)
                  -- node[above=.2em] {(A)}
                  ($(sup-b.base east) + (0, .7em)$);
                \draw[brace]
                  ($(sup-c.base west) + (0, .7em)$)
                  -- node[above=.2em] {(L)}
                  ($(sup-d.base east) + (0, .7em)$);
                \draw[brace80]
                  ($(sup-c.base east) + (0, -.2em)$)
                  -- node[below=.3em, pos=.8] {(M)}
                  ($(sup-b.base west) + (0, -.2em)$);
                \draw[brace80, decoration={mirror}]
                  ($(sup-d.base west) + (0, -.1em)$)
                  -- node[below=.3em, pos=.8] {$\varepsilon > 0$}
                  ($(sup-e.base east) + (0, -.1em)$);
              \end{tikzpicture}
            \\[.6em]
              \big\Downarrow
            \\
              S - \varepsilon < q_k < S + \varepsilon
            \\
              \Updownarrow
            \\
              - \varepsilon < q_k - S < \varepsilon
            \\
              \Updownarrow
            \\
              |q_k - S| < \varepsilon
          \end{array}
        \right.
      }_{\substack{
          \displaystyle \Downarrow
        \\[.3em]
          \displaystyle \lim_{k\to\infty} q_k = S = \sup \{q_k\}
      }}
    &
      \underbracket{
        \forall \varepsilon \exists k_i :
        \substack{
          (\varepsilon > 0) \\[.3em]
          \text{e} \\[.3em]
          (k > k_i)
        }
        {\displaystyle\Bigg\}}
        {\Rightarrow}
        \left\{
          \begin{array}{c}
              \tikzeq{inf-a}{I - \varepsilon}
                <
              \tikzeq{inf-b}{I}
                \le
              \tikzeq{inf-c}{q_k}
                \le
              \tikzeq{inf-d}{q_{k_i}}
                <
              \tikzeq{inf-e}{I + \varepsilon}
              \begin{tikzpicture}[eq-overlay]
                \draw[brace, decoration={mirror}]
                  ($(inf-e.base east) + (0, .7em)$)
                  -- node[above=.2em] {(A)}
                  ($(inf-d.base west) + (0, .7em)$);
                \draw[brace, decoration={mirror}]
                  ($(inf-c.base east) + (0, .7em)$)
                  -- node[above=.2em] {(L)}
                  ($(inf-b.base west) + (0, .7em)$);
                \draw[brace80, decoration={mirror}]
                  ($(inf-c.base west) + (0, -.2em)$)
                  -- node[below=.3em, pos=.8] {(M)}
                  ($(inf-d.base east) + (0, -.2em)$);
                \draw[brace80]
                  ($(inf-b.base east) + (0, -.1em)$)
                  -- node[below=.3em, pos=.8] {$- \varepsilon < 0$}
                  ($(inf-a.base west) + (0, -.1em)$);
              \end{tikzpicture}
            \\[.6em]
              \big\Downarrow
            \\
              I - \varepsilon < q_k < I + \varepsilon
            \\
              \Updownarrow
            \\
              - \varepsilon < q_k - I < \varepsilon
            \\
              \Updownarrow
            \\
              |q_k - I| < \varepsilon
          \end{array}
        \right.
      }_{\substack{
          \displaystyle \Downarrow
        \\[.3em]
          \displaystyle \lim_{k\to\infty} q_k = I = \inf \{q_k\}
      }}
  \end{array}
\]

\textbf{Se a sequência converge,
        a diferença entre elementos consecutivos converge para zero}.
Dado que a sequência de termo geral $q_k$ converge,
então para todo $\varepsilon$ existe um $K$
para o qual $|q_k - L| < \varepsilon$ quando $k \ge K$,
em que $L$ é o limite da sequência.
Essa mesma sequência defasada por um único elemento
irá convergir para o mesmo limite,
bastando fazer um ajuste no índice $k$,
isto é, usando o mesmo $\varepsilon$, sabemos que
$|q_{k+1} - L| < \varepsilon$ quando $k \ge K - 1$.
Quando $k \ge K$, ambas as desigualdades são válidas,
e o limite de $q_{k+1} - q_k$ é zero,
visto que o dobro de um valor arbitrário
continua sendo um valor arbitrário:
\[
  \left.
    \begin{array}{rcl}
        |q_k - L| < \varepsilon
      &\iff&
        - \varepsilon < L - q_k < \varepsilon
      \\
        |q_{k+1} - L| < \varepsilon
      &\iff&
        - \varepsilon < q_{k+1} - L < \varepsilon
    \end{array}
  \right\}
  \;\implies\;
  - 2 \varepsilon < q_{k+1} - q_k < 2 \varepsilon
  \;\iff\;
  |q_{k+1} - q_k| < 2 \varepsilon
\]
