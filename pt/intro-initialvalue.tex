\subsection*{A escolha do valor inicial}

O valor inicial $a_0$ poderia ter sido o próprio $x$,
visto que $x \ge \lfloor \sqrt[n]{x} \rfloor$
para todo $x$ inteiro positivo,
isto é, tal valor satisfaz a restrição definida pelo algoritmo.
Porém, o valor adotado é tipicamente menor que $x$:
\[
  a_0 = \left\lfloor \dfrac{x - 1}{n} \right\rfloor + 1
      = \left\lceil \dfrac{x\vphantom{l}}{n} \right\rceil
\]
Para valores grandes de $x$, esse valor inicial será próximo\footnote{
  Como $\lceil x/n \rceil - 1 < x/n \le \lceil x/n \rceil$
  pela definição da função teto,
  temos $0 \le \lceil x/n \rceil - x/n < 1$.
  Admitindo $n$ constante, a diferença $\lceil x/n \rceil - x/n$
  torna-se menos significativa quanto maior o valor de $x$.
  \[
    \lim_{x\to\infty} \dfrac{\lceil x/n \rceil}{x/n}
    =
    \lim_{x\to\infty} \cancelto{0\;(\text{Limitada}/\infty)}{
                        \dfrac{\lceil x/n \rceil - x/n}{x/n}
                      } +
                      \dfrac{x/n}{x/n}
    = 1
  \]
}
de $x/n$.
A justificativa para a equivalência
entre as duas expressões para o valor inicial
é fornecida após a identificação das propriedades
das funções envolvidas.
Para todo o restante do texto, apenas a função chão é necessária.

O resultado da aplicação da função chão
nunca é maior que o valor sobre o qual ela é aplicada\footnote{
  A definição da função chão é fornecida adiante no texto,
  por hora é suficiente saber que $\lfloor \gamma \rfloor \le \gamma$.
},
o que nos permite escrever:
\[
  \overbrace{\left\lfloor \dfrac{x - 1}{n} \right\rfloor}^{a_0 - 1}
  \le \dfrac{x - 1}{n}
  \quad\iff\quad
  a_0 \le \dfrac{x - 1}{n} + 1
\]

Quando essa escolha de $a_0$
é mais distante da solução
que a adoção do próprio $x$ como valor inicial?
Isto é, quais são os valores de $x$ e $n$
que fazem com que $a_0 > x$?
Para responder a tais perguntas,
adotemos $a_0 > x$ como hipótese,
de forma que $x < a_0 \le \dfrac{x - 1}{n} + 1$.
Logo:
\[
  x < \dfrac{x - 1}{n} + 1
  \quad\iff\quad
  x - 1 < \dfrac{x - 1}{n}
  \quad\stackfirst{n > 0}{\iff}\quad
  n(x - 1) < x - 1
  \quad\iff\quad
  (n - 1)(x - 1) < 0
\]
Como $x$ e $n$ são inteiros positivos,
tal inequação é uma contradição,
o que nos permite dizer que $a_0 \le x$.
Mais genericamente,
podemos agrupar as inequações de números inteiros:
\[
  \lfloor \sqrt[n]{x} \rfloor
  \le
  \left\lfloor \dfrac{x - 1}{n} \right\rfloor + 1
  \le
  x
\]
A primeira parte da inequação, com a raiz $n$-ésima,
ainda precisa ser demonstrada,
mas a segunda parte já está justificada quando $n \ge 1$ e $x \ge 1$.
