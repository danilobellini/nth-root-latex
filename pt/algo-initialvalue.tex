\subsection*{Limitante inferior e valor inicial da sequência}

A desigualdade das médias aritmética e geométrica
pode ser aplicada à média de $n - 1$ ocorrências de $a_k$
e uma ocorrência de $x/a_k^{n-1}$:
\[
  \dfrac{(n-1) a_k + \dfrac{x}{a_k^{n-1}}}{n} \ge
  \sqrt[n]{a_k^{n-1} \dfrac{x}{a_k^{n-1}}}
\]
Aplicando a função chão em cada lado da inequação\footnote{
  Dado $\gamma \le \beta$, sabemos que
  $\lfloor \gamma \rfloor \le \beta$,
  pois $\lfloor \gamma \rfloor \le \gamma$
  pela definição da função chão.
  Aplicando o resultado \eqref{floor-switch-ge}
  à inequação $\lfloor \gamma \rfloor \le \beta$
  fornece $\lfloor \gamma \rfloor \le \lfloor \beta \rfloor$.
  Analogamente, dado $\gamma < \beta$,
  sabemos que $\lfloor \gamma \rfloor < \beta$
  pela definição da função chão,
  e aplicando o resultado \eqref{floor-lt2ge-int}
  obtemos $\lfloor \gamma \rfloor \le \lfloor \beta \rfloor$.
  Resumidamente,
  é sempre possível aplicar a função chão
  nos dois lados de uma inequação,
  mas a inequação do resultado deverá ser não-estrita.
  O mesmo resultado pode ser interpretado
  como consequência do fato
  de que a função chão é monotônica não-decrescente,
  mas não é estritamente crescente.
},
temos:
\[
  a_{k+1} \ge \lfloor \sqrt[n]{x} \rfloor
\]
De forma que, basta termos $a_0 > 0$
para garantirmos $a_k \ge \lfloor \sqrt[n]{x} \rfloor$ para $k > 0$.
Mas seria interessante adotar uma condição inicial
que também mantivesse essa propriedade.
Para isso, suponha que $a_s = 1$, com isso teríamos:
\[
  a_{s+1}
  = \left\lfloor \dfrac{n - 1 + x}{n} \right\rfloor
  = \left\lfloor \dfrac{x - 1}{n} + 1 \right\rfloor
  \stackfirst{\eqref{floor-int}}{=}
    \left\lfloor \dfrac{x - 1}{n} \right\rfloor + 1
\]
Isso significa que:
\[
  \left\lfloor \dfrac{x - 1}{n} \right\rfloor + 1
  \ge \lfloor \sqrt[n]{x} \rfloor
\]
Ou seja, caso adotemos como condição inicial:
\[
  a_0 = \left\lfloor \dfrac{x - 1}{n} \right\rfloor + 1
\]
Garantimos $a_k \ge \lfloor \sqrt[n]{x} \rfloor$
para todo $k \in \mathds{N}$.
