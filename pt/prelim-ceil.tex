\subsection*{Função teto e sua relação com a função chão}

A função teto pode ser definida
como a função $\lceil \cdot \rceil : \mathds{R} \to \mathds{Z}$
que satisfaz:
\begin{equation}\tag{Teto}
  \lceil \gamma \rceil - 1 < \gamma \le \lceil \gamma \rceil
\end{equation}
Isto é,
$\lceil \gamma \rceil$ é o menor inteiro maior ou igual a $\gamma$.

Pelas definições das funções chão e teto,
quando $\gamma$ é inteiro, temos:
\begin{equation}\label{ceil-floor-z}
  \lfloor \gamma \rfloor = \gamma = \lceil \gamma \rceil
  \quad \forall \gamma \in \mathds{Z}
\end{equation}
E, para o cenário complementar, quando $\gamma \notin \mathds{Z}$,
podemos explicitar que as inequações das definições não são estritas,
isto é, $\lfloor \gamma \rfloor < \gamma < \lfloor \gamma \rfloor + 1$
e $\lceil \gamma \rceil - 1 < \gamma < \lceil \gamma \rceil$.
Pela propriedade transitiva, podemos obter duas novas inequações:
$\lfloor \gamma \rfloor < \lceil \gamma \rceil$
e $\lceil \gamma \rceil - 1 < \lfloor \gamma \rfloor + 1$.
O resultado \eqref{int+1}
aplicado a cada uma dessas duas inequações nos fornece
$\lfloor \gamma \rfloor + 1 \le \lceil \gamma \rceil$ e
$\lceil \gamma \rceil \le \lfloor \gamma \rfloor + 1$, respectivamente.
Isso justifica a igualdade
$\lceil \gamma \rceil = \lfloor \gamma \rfloor + 1$,
que pode ser reescrita como:
\begin{equation}\label{ceil-floor-notz}
  \lceil \gamma \rceil - \lfloor \gamma \rfloor = 1
  \quad \forall \gamma \in \mathds{R} \setminus \mathds{Z}
\end{equation}
Podemos aplicar esses últimos resultados
\eqref{ceil-floor-z} e \eqref{ceil-floor-notz}
em meio a provas de expressões
que envolvam simultaneamente as funções chão e teto,
bastando utilizar o particionamento entre inteiros e não-inteiros
para provar cada caso separadamente.
Esse recurso será usado no que segue.

Sejam $x \in \mathds{Z}$ e $n \in \mathds{N}^*$.
Quando $x/n \in \mathds{Z}$, podemos escrever:
\[
    \left\lfloor \dfrac{x - 1}{n} \right\rfloor =
    \left\lfloor \dfrac{x}{n} - \dfrac{1}{n} \right\rfloor
      \stackfirst{\text{\eqref{floor-int}}}{=}
    \dfrac{x}{n} + \left\lfloor - \dfrac{1}{n} \right\rfloor =
    \dfrac{x}{n} - 1
  \quad\implies\quad
    \left\lfloor \dfrac{x - 1}{n} \right\rfloor + 1 =
    \dfrac{x}{n}
      \stackfirst{(\mathds{Z})}{=}
    \left\lceil \dfrac{x\vphantom{l}}{n} \right\rceil
\]
Pela aplicação da propriedade
descrita na equação \eqref{floor-int}
e pelo fato de que $\lfloor -1/n \rfloor = -1$,
visto que $-1 \le -1/n < 0$.
Por outro lado, quando $x/n \notin \mathds{Z}$,
partindo da definição da função chão,
já descartando a igualdade por ela ser exclusiva dos inteiros,
e utilizando os resultados \eqref{int+1} e \eqref{floor-switch-ge},
podemos escrever:
\begin{align*}
    \left\lfloor \dfrac{x\vphantom{l}}{n} \right\rfloor
    < \dfrac{x\vphantom{l}}{n}
  &\quad\stackrel{\rule[-.3em]{0pt}{0pt}n > 0}{\iff}\quad
    n \left\lfloor \dfrac{x\vphantom{l}}{n} \right\rfloor < x
  \\&\quad\iff\quad
    0 < x - n \left\lfloor \dfrac{x\vphantom{l}}{n} \right\rfloor
  \\&\quad\stackrel{\rule[-.3em]{0pt}{0pt}\eqref{int+1}}{\iff}\quad
    1 \le x - n \left\lfloor \dfrac{x\vphantom{l}}{n} \right\rfloor
  \\&\quad\iff\quad
    n \left\lfloor \dfrac{x\vphantom{l}}{n} \right\rfloor \le x - 1
  \\&\quad\stackrel{\rule[-.3em]{0pt}{0pt}n > 0}{\iff}\quad
    \left\lfloor \dfrac{x\vphantom{l}}{n} \right\rfloor
    \le \dfrac{x - 1}{n}
  \\&\quad\stackrel{\rule[-.3em]{0pt}{0pt}\eqref{floor-switch-ge}}{\iff}\quad
    \left\lfloor \dfrac{x\vphantom{l}}{n} \right\rfloor
    \le \left\lfloor \dfrac{x - 1}{n} \right\rfloor
\end{align*}
Podemos também aproveitar o resultado \eqref{floor-lt2ge-int}
e a definição da função chão
$\left\lfloor \dfrac{x - 1}{n} \right\rfloor \le \dfrac{x - 1}{n}$ em:
\[
    -1 < 0
  \quad\iff\quad
    x - 1 < x
  \quad\stackfirst{n > 0}{\iff}\quad
    \dfrac{x - 1}{n} < \dfrac{x}{n}
  \quad\stackfirst{\text{(Chão)}}{\implies}\quad
    \left\lfloor \dfrac{x - 1}{n} \right\rfloor < \dfrac{x}{n}
  \quad\stackfirst{\eqref{floor-lt2ge-int}}{\iff}\quad
    \left\lfloor \dfrac{x - 1}{n} \right\rfloor
    \le \left\lfloor \dfrac{x\vphantom{l}}{n} \right\rfloor
\]
Ou seja,
$\left\lfloor \dfrac{x - 1}{n} \right\rfloor =
 \left\lfloor \dfrac{x\vphantom{l}}{n} \right\rfloor$
quando $x/n \notin \mathds{Z}$.
Mas nesse caso também sabemos que
$\left\lceil \dfrac{x\vphantom{l}}{n} \right\rceil -
 \left\lfloor \dfrac{x\vphantom{l}}{n} \right\rfloor = 1$.
Logo:
\[
  \left\lceil \dfrac{x\vphantom{l}}{n} \right\rceil
  = 1 + \left\lfloor \dfrac{x\vphantom{l}}{n} \right\rfloor
  = 1 + \left\lfloor \dfrac{x - 1}{n} \right\rfloor
\]
Que é o mesmo resultado encontrado quando $x/n \in \mathds{Z}$.
Com isso, concluímos que:
\begin{equation}\label{ceil-x-over-n-as-floor}
  \left\lceil \dfrac{x\vphantom{l}}{n} \right\rceil
  = \left\lfloor \dfrac{x - 1}{n} \right\rfloor + 1
  \quad \forall x \in \mathds{Z}, \; \forall n \in \mathds{N}^*
\end{equation}
Esse valor é precisamente a equação da condição inicial
adotada para o algoritmo.
