\subsection*{O método de Newton-Raphson}

O método de Newton-Raphson consiste em um procedimento numérico
para a obtenção de uma sequência que, caso seja convergente,
converge para uma raiz de uma dada função contínua e derivável
$f:\mathds{R}\to\mathds{R}$, isto é,
dado $\alpha_k \in \mathds{R}$ o método fornece
um equacionamento para $\alpha_{k+1}$,
e, se a sequência é definida para todo $k \in \mathds{N}$ e convergente,
$f(\alpha_k) \to 0$ quando $k \to \infty$.
Especificamente,
a regra de iteração do método de Newton-Raphson
consiste no uso da reta tangente à função
no ponto $\big(\alpha_k, f(\alpha_k)\big)$,
definindo a nova abscissa $\alpha_{k+1}$
como a raiz dessa aproximação linear da função.
Esse processo está ilustrado na figura~\ref{fig:newton-raphson}.

\begin{figure}[h!]
  \centering
  \begin{tikzpicture}[
    domain=-.5:15.3,
    scale=.7,
    samples=100,
    axis/.style={very thick, >=stealth', ->},
    grid/.style={thin, color=gray},
    func/.style={thick, color=darkgray},
    approx/.style={thin, color=gray, >=stealth', ->},
  ]

  \draw[grid, dotted, step=1] (0, 0) grid (16, 8);
  \draw[axis] (0, -.5) -- (0, 8.5) node[above] {$f(\alpha)$};
  \draw[axis] (-.5, 0) -- (16.5, 0) node[right] {$\alpha$};
  \draw[func] plot (\x, {(\x / 10) ^ 5 - .2});

  \draw[grid, dashed]
    (15, 7.39375)
      -- (15, 0)
      node[below, color=black] {$\alpha_0$}
    (12.079012345679013, 2.3713259083991125)
      -- (12.079012345679013, 0)
      node[below, color=black] {$\alpha_1$}
    (9.851113126088102, 0.7277405339275622)
      -- (9.851113126088102, 0)
      node[below, color=black] {$\alpha_2$}
    (8.305626200813814, 0.1952409256907805)
      -- (8.305626200813814, 0)
      node[below, color=black] {$\alpha_3$}
    (7.485064339688552, 0.034951211574533014)
      -- (7.485064339688552, 0)
      node[below, color=black] {$\alpha_4$};

  \draw[approx, solid]
    (15, 7.39375)
      -- (12.079012345679013, 0);
  \draw[approx, solid]
    (12.079012345679013, 2.3713259083991125)
      -- (9.851113126088102, 0);
  \draw[approx, solid]
    (9.851113126088102, 0.7277405339275622)
      -- (8.305626200813814, 0);
  \draw[approx, solid]
    (8.305626200813814, 0.1952409256907805)
      -- (7.485064339688552, 0);

  \foreach \x/\xtext in {1, 2, 3, 4, 5, 6, 7, 8, 9, 10,
                         11, 12, 13, 14, 15, 16}
    \draw[shift={(\x, 0)}] (0pt, 2pt) -- (0pt, -2pt) node[below] {};
  \foreach \y in {1, 2, 3, 4, 5, 6, 7, 8}
    \draw[shift={(0, \y)}] (2pt, 0pt) -- (-2pt, 0pt) node[left] {};

\end{tikzpicture}

  \caption{Ilustração do método de Newton-Raphson}
  \label{fig:newton-raphson}
\end{figure}

Uma reta não-vertical pode ser escrita como uma função,
a qual pode ser encontrada
a partir de seu coeficiente angular e de um ponto conhecido.
Seja $g_k(\alpha):\mathds{R}\to\mathds{R}$ a reta
que possui um fragmento representado na figura~\ref{fig:newton-raphson}
como uma flecha unindo os pontos
$\big(\alpha_k, f(\alpha_k)\big)$ e $\big(\alpha_{k+1}, 0\big)$.
Sabemos que a variação média e a variação instantânea de uma reta
são sempre iguais, isto é,
$\Delta g_k / \Delta \alpha$ é uma constante,
e seu valor é igual ao da derivada $g'_k(\alpha) = c_k$,
a qual também é constante para toda a reta.
Explicitamente:
\[
  c_k = g'_k(\alpha)
  =
    \lim_{h \to 0}
      \dfrac{g_k(\alpha + h) - g_k(\alpha)}{h}
  =
    \dfrac{\Delta g_k}{\Delta \alpha}
  =
    \dfrac{g_k(\alpha_k) - g_k(\alpha_{k+1})}{\alpha_k - \alpha_{k+1}}
\]
De onde é possível obter a equação da reta:
\[
  g_k(\alpha_k) - g_k(\alpha_{k+1}) =
  c_k(\alpha_k - \alpha_{k+1})
\]

Como o ponto $\big(\alpha_{k+1}, 0\big)$ pertence à reta.
sabemos que a raiz da aproximação linear é $g_k(\alpha_{k+1}) = 0$.
Também sabemos que o ponto de tangência
pertence tanto à função $f(\alpha)$
quanto à reta tangente $g_k(\alpha)$, isto é,
$\big(\alpha_k, f(\alpha_k)\big) = \big(\alpha_k, g_k(\alpha_k)\big)$,
ou simplesmente $g_k(\alpha_k) = f(\alpha_k)$.
Além disso, a inclinação nesse ponto de tangência
deve ser a mesma para ambas as funções $f(\alpha)$ e $g_k(\alpha)$,
de forma que $f'(\alpha_k) = g_k'(\alpha_k) = c_k$.
Podemos aplicar essas informações na equação da reta:
\[
    \underbrace{g_k(\alpha_k)}_{f(\alpha_k)}
    - \cancelto{0}{g_k(\alpha_{k+1})}
  =
    \underbracket[1.1pt]{c_k}_{\mathclap{f'(\alpha_k)}}
    (\alpha_k - \alpha_{k+1})
  \quad\stackfirst{f'(\alpha_k) \ne 0}{\implies}\quad
  \dfrac{f(\alpha_k)}{f'(\alpha_k)} = \alpha_k - \alpha_{k+1}
\]
O que nos permite obter a recorrência
que define a sequência do método de Newton-Raphson:
\[\tag{NR}
  \alpha_{k+1} = \alpha_k - \dfrac{f(\alpha_k)}{f'(\alpha_k)}
\]
