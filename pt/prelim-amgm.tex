\subsection*{Desigualdade entre as médias aritmética e geométrica}

Dados $n \in \mathds{N}^*$ e números reais $y_i \ge 0$,
a desigualdade entre as médias aritmética e geométrica
consiste em dizer que
\emph{a média aritmética é maior ou igual à média geométrica}, isto é:
\[\tag{MA $\ge$ MG}
    \overbrace{
      \dfrac{1}{n} \sum_{i=1}^n y_i
    }^{\mathclap{\text{Média aritmética}}}
  \ge
    \underbrace{
      \sqrt[n]{\prod_{i=1}^n y_i}
    }_{\mathclap{\text{Média geométrica}}}
\]
Para $n = 1$ esse resultado é imediato,
visto que ambas as médias de um único elemento
são iguais a esse próprio elemento.
Para $n = 2$, também sabemos que a desigualdade é verdadeira, pois:
\begin{align*}
    (y_1 - y_2)^2 \ge 0
  &\iff
    y_1^2 - 2 y_1 y_2 + y_2^2 \ge 0
  \\&\iff
    y_1^2 + 2 y_1 y_2 + y_2^2 \ge 4 y_1 y_2
  \\&\iff
    (y_1 + y_2)^2 \ge 4 y_1 y_2
  \\&\stackrel{\rule[-.3em]{0pt}{0pt}y_i \ge 0}{\iff}\vphantom{\int}
    y_1 + y_2 \ge 2 \sqrt{y_1 y_2}
  \\&\iff
    \dfrac{y_1 + y_2}{2} \ge \sqrt{y_1 y_2}
  \tag{B}
\end{align*}

É possível provar essa desigualdade para qualquer $n \in \mathds{N}^*$
por meio de duas induções em direções opostas.
Para a primeira indução, admita que $n = 1$ ou $n = 2$ é o caso base.
Se a desigualdade vale para $n$ elementos,
ela também valerá para $2n$ elementos, pois:
\[
    \dfrac{1}{2n} \sum_{i=1}^{2n} y_i
  =
    \dfrac{1}{2}
    \left(
      \dfrac{1}{n} \sum_{i=1}^{n} y_i +
      \dfrac{1}{n} \sum_{i=n+1}^{2n} y_i
    \right)
  \stackfirst{\text{(H)}}{\ge}
    \dfrac{1}{2}
    \left(\vphantom{\sqrt{\prod_{i=1}^n}}\right.
      \underbrace{\sqrt[n]{\prod_{i=1}^n y_i}}_{z_1} +
      \underbrace{\sqrt[n]{\prod_{i=n+1}^{2n} y_i}}_{z_2}
    \left.\vphantom{\sqrt{\prod_{i=1}^n}}\right)
  \stackfirst{\text{(B)}}{\ge}
    \underbrace{
      \sqrt{ \left( \sqrt[n]{\prod_{i=1}^n y_i} \right)
             \left( \sqrt[n]{\prod_{i=n+1}^{2n} y_i} \right) }
    }_{\sqrt{z_1 z_2}}
\]\[
    \dfrac{1}{2n} \sum_{i=1}^{2n} y_i
  \ge
    \sqrt{ \sqrt[n]{ \left( \prod_{i=1}^n y_i \right)
                     \left( \prod_{i=n+1}^{2n} y_i \right) } }
  =
    \sqrt[2n]{\prod_{i=1}^{2n} y_i}
\]
Em que (H) denota o uso da \emph{hipótese} dessa indução
para os $2$ conjuntos,
e (B) denota o uso do resultado encontrado para $n = 2$
aplicado aos valores $z_i$ indicados.
Isso prova que a desigualdade entre as médias aritmética e geométrica
vale para todas as potências de $2$, isto é,
para $n = 2^m$, $m \in \mathds{N}^*$.
Com isso já sabemos que não há um valor máximo
para o qual a desigualdade vale\footnote{
  Se adotarmos $w = \lceil\log_2 n\rceil$, sabemos que $n \le 2^w$
  pelo fato de todas as funções envolvidas serem não-decrescentes.
  Para qualquer $n$,
  podemos construir uma potência de $2$ maior ou igual a $n$,
  e $2^w$ é a menor potência de $2$ que satisfaz esse requisito.
},
e podemos realizar uma indução no sentido inverso,
na qual podemos adotar qualquer potência de $2$ como novo caso base.

Adotando $n = 2^m$ para qualquer $m \in \mathds{N}^*$ como caso base,
e admitindo como hipótese de indução
que a desigualdade entre as médias aritmética e geométrica
vale para $n$ elementos,
queremos mostrar que ela também valerá para $n-1$ elementos.
Para isso, podemos considerar que temos apenas $n-1$ elementos $y_i$,
e inserimos o $n$-ésimo elemento como a média aritmética dos demais.
Isto é, considere:
\[
  y_n = \dfrac{1}{n-1} \sum_{i=1}^{n-1} y_i
\]
Isso já nos obriga a utilizar $n \ge 2$
como restrição deste processo,
para evitar uma divisão por zero.
Utilizando esse $n$-ésimo elemento, a média aritmética torna-se:
\[
    \dfrac{1}{n} \sum_{i=1}^{n} y_i
  =
    \dfrac{1}{n} \left( y_n + \sum_{i=1}^{n-1} y_i \right)
  =
    \dfrac{y_n}{n} + \dfrac{1}{n} \sum_{i=1}^{n-1} y_i
  =
    \dfrac{y_n}{n} + \dfrac{(n - 1) y_n}{n}
  =
    y_n
  \stackfirst{\text{(H)}}{\ge}
    \sqrt[n]{\prod_{i=1}^n y_i}
  =
    \sqrt[n]{y_n \prod_{i=1}^{n-1} y_i}
\]
Em que o (H) denota o uso da hipótese dessa indução
com base na expressão mais à esquerda,
utilizando o fato de que os passos intermediários são todos iguais.
Caso $y_n > 0$, temos:
\[
    y_n \ge \sqrt[n]{y_n \prod_{i=1}^{n-1} y_i}
  \;\implies\;
    y_n^n \ge y_n \prod_{i=1}^{n-1} y_i
  \;\stackfirst{y_n > 0}{\implies}\;
    y_n^{n-1} \ge \prod_{i=1}^{n-1} y_i
  \;\implies\;
    y_n \ge \sqrt[n-1]{\prod_{i=1}^{n-1} y_i}
\]
Que é exatamente a desigualdade entre as médias aritmética e geométrica
para $n - 1$ elementos.
Caso $y_n = 0$, todos os elementos $y_i$ precisarão ser iguais a zero,
visto que esta é uma soma de números não-negativos,
e isso resulta na igualdade das duas médias (ambas iguais a zero),
o que também satisfaz
a desigualdade entre as médias aritmética e geométrica.

Dado que o caso base pode ser arbitrariamente grande,
essa segunda indução mostra que a desigualdade é válida
para qualquer $n \ge 2$,
``preenchendo as lacunas'' dos números que não são potências de $2$.
