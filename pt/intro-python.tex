\subsection*{Implementação do algoritmo em Python}

O algoritmo proposto pode ser escrito em Python
utilizando uma \emph{assignment expression}\footnote{
  Expressão realizando uma atribuição.
  Essa sintaxe com o operador ``\texttt{:=}'' (\emph{walrus})
  está disponível a partir do Python 3.8.
}:

\begin{center}
  \begin{minipage}{7cm}
    \inputminted{python}{nth_root.py}
  \end{minipage}
\end{center}

A menos de espaços em branco,
as $3$ últimas linhas desse algoritmo
formam exatamente o fragmento de código que consta na PEP 572.
A primeira linha apenas define um bloco como uma \emph{função},
no sentido em que essa palavra é usada
no contexto de linguagens de programação.
A segunda linha define um valor inicial específico,
uma novidade do presente texto, a qual não consta na PEP 572.
