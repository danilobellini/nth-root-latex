\section*{Considerações preliminares}

Para a apropriada compreensão
das análises e provas realizadas no decorrer deste texto,
são necessários conhecimentos sobre matemática elementar, conjuntos,
potenciação, raízes, valor absoluto, somatórios, produtórios,
relações de ordem e suas propriedades,
provas (direta, contraposição, indução, contradição,
        separação em casos),
bem como um mínimo de cálculo diferencial em uma variável real.
O restante dos requisitos matemáticos para a compreensão do texto
são apresentados a seguir.

A PEP 572 fornece duas \emph{dicas}
para provar a corretude do algoritmo:
\textsc{(i)} a desigualdade entre as médias aritmética e geométrica,
e \textsc{(ii)} as propriedades do aninhamento de funções chão.
Sobre o segundo item,
não há nenhuma informação sobre quais propriedades são essas,
exceto que elas não são triviais.
Explicações acerca desses assuntos,
incluindo as provas das propriedades necessárias
ou de alguma forma vinculadas ao algoritmo em análise,
encontram-se nesta seção do texto.

O restante desta seção está dividido por assunto.
Analisaremos brevemente assuntos relacionados
aos números positivos,
às funções chão e teto,
à desigualdade entre as médias aritmética e geométrica,
e à convergência de sequências de números reais.
