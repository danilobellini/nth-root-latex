\subsection*{Sinal do passo e condição de parada}

Suponha $a_k = \sqrt[n]{x}$.
Como $a_k \in \mathds{N}$,
também sabemos que $\sqrt[n]{x} = \lfloor \sqrt[n]{x} \rfloor$.
Elevando a $n$ cada lado da igualdade, obtemos $a_k^n = x$,
o que significa que $p_{k+1} = 0$,
ou seja, a continuação da sequência será constante,
igual a $\sqrt[n]{x}$.
O resultado $a_k$ já é o valor da raiz desejada,
não sendo necessário continuar o processo iterativo.

Suponha $a_k > \sqrt[n]{x}$.
Nesse caso, $a_k^n > x$, e portanto $x - a_k^n < 0$,
i.e., o numerador de $p_{k+1}$ é negativo.
Logo $p_{k+1} \le -1$, pois seu denominador é positivo
e o resultado da função chão aplicada a qualquer número negativo
é sempre um número inteiro negativo.

Suponha $\lfloor \sqrt[n]{x} \rfloor \le a_k < \sqrt[n]{x}$.
Obviamente, isso somente poderá ocorrer
se $\lfloor \sqrt[n]{x} \rfloor \ne \sqrt[n]{x}$.
Pela definição da função chão, temos
$\lfloor\sqrt[n]{x}\rfloor \le a_k < \lfloor\sqrt[n]{x}\rfloor + 1$,
o que nos obriga a dizer que $a_k = \lfloor\sqrt[n]{x}\rfloor$
pelo fato de $a_k$ ser sempre um número inteiro.
Nesse contexto, temos:
\[
  \underbrace{\lfloor \sqrt[n]{x} \rfloor}_{a_k} < \sqrt[n]{x}
  \quad\implies\quad
  a_k^n < x
  \quad\implies\quad
  x - a_k^n > 0
  \quad\implies\quad
  \dfrac{x - a_k^n}{na_k^{n-1}} > 0
  \quad\implies\quad
  \underbrace{
    \left\lfloor \dfrac{x - a_k^n}{na_k^{n-1}} \right\rfloor
  }_{p_{k+1}} \ge 0
\]

Resumidamente, o que está escrito acima pode ser escrito como:
\[
\begin{cases}
  a_k > \sqrt[n]{x} &\implies\;\; p_{k+1} < 0 \\
  a_k \le \sqrt[n]{x} &\implies\;\; p_{k+1} \ge 0
\end{cases}
\]

Como $a_k \le \sqrt[n]{x}$ na realidade significa
$a_k = \lfloor \sqrt[n]{x} \rfloor$,
podemos considerar que o critério de parada é $p_{k+1} \ge 0$.
Em outras palavras,
\emph{o processo iterativo deve continuar
      apenas enquanto $a_k$ estiver diminuindo},
e quando parar de diminuir,
\emph{o valor mínimo encontrado é o próprio resultado desejado}.
Como estamos lidando apenas com valores inteiros,
isso também significa que o resultado é obtido
após um número finito de iterações.

Há ainda outra forma de consultar essa condição de parada,
utilizando propriedades da função chão:
\[
  \left\lfloor \dfrac{x - a_k^n}{na_k^{n-1}} \right\rfloor < 0
  \;\iff\;
  \dfrac{x - a_k^n}{na_k^{n-1}} < 0
  \;\stackfirst{a_k > 0}{\iff}\;
  x - a_k^n < 0
  \;\iff\;
  a_k^n > x
  \;\stackfirst{a_k > 0}{\iff}\;
  a_k > \dfrac{x}{a_k^{n-1}}
  \;\stackfirst{a_k \in \mathds{N}}{\iff}\;
  a_k > \left\lfloor \dfrac{x}{a_k^{n-1}} \right\rfloor
\]
\[
  \left\lfloor \dfrac{x - a_k^n}{na_k^{n-1}} \right\rfloor \ge 0
  \;\iff\;
  \dfrac{x - a_k^n}{na_k^{n-1}} \ge 0
  \;\stackfirst{a_k > 0}{\iff}\;
  x - a_k^n \ge 0
  \;\iff\;
  a_k^n \le x
  \;\stackfirst{a_k > 0}{\iff}\;
  a_k \le \dfrac{x}{a_k^{n-1}}
  \;\stackfirst{a_k \in \mathds{N}}{\iff}\;
  a_k \le \left\lfloor \dfrac{x}{a_k^{n-1}} \right\rfloor
\]

Como $d_{k+1} = \left\lfloor \dfrac{x}{a_k^{n-1}} \right\rfloor$,
a condição de parada é $a_k \le d_{k+1}$:
\[
\begin{cases}
  a_k > \sqrt[n]{x} &\implies\;\; a_k > d_{k+1} \\
  a_k \le \sqrt[n]{x} &\implies\;\; a_k \le d_{k+1}
\end{cases}
\]
O que corresponde à descrição do algoritmo.
