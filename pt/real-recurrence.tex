\subsection*{Recorrência em $\mathds{R}$ para encontrar $\sqrt[n]{x}$}

Dados números inteiros positivos $x$ e $n$,
seja $f:\mathds{R}^+\to\mathds{R}$ a função polinomial:
\begin{equation}
  f(\alpha) = \alpha^n - x
\end{equation}
A qual podemos notar que satisfaz $f(\sqrt[n]{x}) = 0$.
Nosso objetivo é encontrar uma sequência
que convirja para esse zero dessa função,
o que corresponde a encontrar $\alpha = \sqrt[n]{x}$.

A derivada de $f$ com relação a $\alpha$ é:
\begin{equation}\label{real-derivative}
  f'(\alpha) = n \alpha^{n - 1}
\end{equation}
Que é estritamente positiva para $\alpha$ positivo,
o que significa que $f(\alpha)$ é monotônica crescente
em todo o domínio $\alpha > 0$.
Como $f(\alpha) \to -x < 0$ quando $\alpha \to 0$,
$f(\alpha) \to \infty > 0$ quando $\alpha \to \infty$,
e a função $f$ é contínua,
podemos dizer que $\sqrt[n]{x}$
é a única raiz real positiva dessa função\footnote{
  Outra forma de identificar que essa raiz é única
  consiste em lembrar que
  \textsc{(i)} um polinômio de grau $n$
  pode ser escrito como uma constante
  multiplicada pelo produtório de $n$ termos $(z - z_m)$
  para cada uma de suas $z_m$ raízes complexas,
  e que \textsc{(ii)} as $n$ raízes de $z^n - 1$,
  chamadas de \emph{raízes da unidade},
  são $z_m = e^{i 2 \pi m / n}$,
  para todos os valores $m$ de $0$ a $n - 1$,
  em que $i$ é a unidade imaginária que satisfaz $i^2 = -1$,
  e $e^{i\theta} = \cos\theta + i\operatorname{sen}\theta$.
  Esse resultado pode ser interpretado como
  uma identificação de arcos no círculo unitário do plano complexo
  para dividir $m$ voltas completas em $n$ partes iguais.
  Com isso, as $n$ raízes de $f(\alpha)$
  são da forma $\sqrt[n]{x} \; e^{i 2 \pi m / n}$
  para os diferentes valores inteiros de $m$,
  e para $m = 0$ temos a única raiz real positiva.
}.

Aplicando os valores de $f(\alpha)$ e $f'(\alpha)$
à equação da sequência do método de Newton-Raphson:
\[
  \alpha_{k+1} =
    \alpha_k - \dfrac{\alpha_k^n - x}{n \alpha_k^{n-1}}
  =
    \dfrac{n \alpha_k^n - \alpha_k^n + x}{n \alpha_k^{n-1}}
  =
    \dfrac{(n-1) \alpha_k^n + x}{n \alpha_k^{n-1}}
  =
    \dfrac{(n-1) \alpha_k + \dfrac{x}{\alpha_k^{n-1}}}{n}
\]
Escrevendo de outra forma:
\begin{equation}\label{real-recurrence}
  \alpha_{k+1} = \dfrac{(n-1) \alpha_k + \delta_{k+1}}{n},
  \quad\text{onde}\quad
  \delta_{k+1} = \dfrac{x}{\alpha_k^{n-1}}
\end{equation}
O que é bastante similar à recorrência
proposta anteriormente para o algoritmo.
Esse novo equacionamento não possui as funções chão
que tornavam todos os resultados intermediários naturais/inteiros,
mas resta ainda avaliar se essa sequência converge.
