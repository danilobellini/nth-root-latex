\subsection*{Números positivos}

A fim de evitar possíveis ambiguidades
relativas à diferença de notação para diferentes autores,
vale destacar que, no contexto do presente texto,
o conjunto $\mathds{N}$ dos números naturais inclui o zero,
o conjunto dos números inteiros positivos é
$\mathds{N}^* = \mathds{N} \setminus \{0\}$,
e o conjunto dos números reais positivos é
$\mathds{R}^+ = \{x: x > 0 \text{ e } x \in \mathds{R}\}$.

Um resultado relevante acerca da comparação de números inteiros
consiste em usar o ``passo'' unitário
para converter entre desigualdades estritas e não-estritas.
Dado $m \in \mathds{Z}$, sabemos que $m > 0 \iff m \ge 1$,
isto é, $1$ é o ínfimo do conjunto dos inteiros positivos.
Escrevendo $m$ como uma subtração de inteiros $m = \ell - k$, temos:
\begin{equation}\label{int+1}
  k < \ell \iff k + 1 \le \ell,
  \quad \forall k \in \mathds{Z}, \; \forall \ell \in \mathds{Z}
\end{equation}
