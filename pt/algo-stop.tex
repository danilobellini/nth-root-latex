\subsection*{Sinal do passo e condição de parada}

Utilizando os resultados acerca da função chão
apresentados em \eqref{floor-switch-lt} e \eqref{floor-switch-ge},
podemos relacionar o sinal dos valores de $a_k$, $p_{k+1}$ e $d_{k+1}$:
\[
  \overbrace{
    \left\lfloor \dfrac{x - a_k^n}{na_k^{n-1}} \right\rfloor
  }^{p_{k+1}} < 0
  \;\stackfirst{\eqref{floor-switch-lt}}{\iff}\;
  \dfrac{x - a_k^n}{na_k^{n-1}} < 0
  \;\stackfirst{a_k > 0}{
      \underset{n > 0\vphantom{\int}}{\iff}
    }\;
  x - a_k^n < 0
  \;\iff\;
  \overset{
    \mathclap{\substack{
      \displaystyle a_k > \sqrt[n]{x} \\[.5em]
      \displaystyle \Big\Updownarrow \\[.5em]
    }}
  }{a_k^n > x}
  \;\stackfirst{a_k > 0}{\iff}\;
  a_k > \dfrac{x}{a_k^{n-1}}
  \;\underset{\eqref{floor-switch-lt}}{
      \stackfirst{a_k \in \mathds{N}}{\iff}
    }\;
  a_k > \overbrace{
          \left\lfloor \dfrac{x}{a_k^{n-1}} \right\rfloor
        }^{d_{k+1}}
\]
\[
  \underbrace{
    \left\lfloor \dfrac{x - a_k^n}{na_k^{n-1}} \right\rfloor
  }_{p_{k+1}} \ge 0
  \;\stackfirst{\eqref{floor-switch-ge}}{\iff}\;
  \dfrac{x - a_k^n}{na_k^{n-1}} \ge 0
  \;\stackfirst{a_k > 0}{
      \underset{n > 0\vphantom{\int}}{\iff}
    }\;
  x - a_k^n \ge 0
  \;\iff\;
  \underset{
    \mathclap{\substack{\\[.4em]
      \displaystyle \Big\Updownarrow \\[.5em]
      \displaystyle a_k \le \sqrt[n]{x}
    }}
  }{a_k^n \le x}
  \;\stackfirst{a_k > 0}{\iff}\;
  a_k \le \dfrac{x}{a_k^{n-1}}
  \;\underset{\eqref{floor-switch-ge}}{
      \stackfirst{a_k \in \mathds{N}}{\iff}
    }\;
  a_k \le \underbrace{
            \left\lfloor \dfrac{x}{a_k^{n-1}} \right\rfloor
          }_{d_{k+1}}
\]
Tal resultado pode ser resumido como:
\[
  \begin{array}{lcccr}
    p_{k+1}   < 0 &\iff& a_k   > \sqrt[n]{x} &\iff& a_k   > d_{k+1} \\
    p_{k+1} \ge 0 &\iff& a_k \le \sqrt[n]{x} &\iff& a_k \le d_{k+1}
  \end{array}
\]
Resta ainda entender melhor o cenário em que $a_k \le \sqrt[n]{x}$,
bem como a forma com a qual esse resultado
se relaciona com a corretude do algoritmo e o critério de parada.

Suponha $\lfloor \sqrt[n]{x} \rfloor \le a_k < \sqrt[n]{x}$.
Pela definição da função chão, temos
$\sqrt[n]{x} < \lfloor\sqrt[n]{x}\rfloor + 1$.
Juntar as desigualdades fornece
$\lfloor\sqrt[n]{x}\rfloor \le a_k < \lfloor\sqrt[n]{x}\rfloor + 1$,
o que nos obriga a dizer que $a_k = \lfloor\sqrt[n]{x}\rfloor$
pelo fato de $a_k$ ser sempre um número inteiro
$a_k = \lfloor a_k \rfloor$,
e pelas últimas desigualdades consistirem
na própria definição da função chão.
Isto é:
\[
  \lfloor \sqrt[n]{x} \rfloor \le a_k < \sqrt[n]{x}
  \quad\implies\quad
  a_k = \lfloor \sqrt[n]{x} \rfloor
\]
Por outro lado, suponha $a_k = \sqrt[n]{x}$.
Como $a_k \in \mathds{N}^*$,
também sabemos que $\sqrt[n]{x} = \lfloor \sqrt[n]{x} \rfloor$.
Isso significa que
quando $\lfloor \sqrt[n]{x} \rfloor \le a_k \le \sqrt[n]{x}$,
então $a_k = \lfloor \sqrt[n]{x} \rfloor$.
Como a desigualdade entre as médias aritmética e geométrica
anteriormente nos indicou que
$a_{k+1} \ge \lfloor \sqrt[n]{x} \rfloor$
podemos reescrever
a relação entre o sinal dos valores de $a_k$, $p_{k+1}$ e $d_{k+1}$
como uma variante mais restrita,
admindo que $k > 0$ ou $a_0 \ge \lfloor \sqrt[n]{x} \rfloor$:
\[
  \begin{array}{lcccr}
      p_{k+1} < 0
    &\iff&
      a_k > \sqrt[n]{x}
    &\iff&
      a_k > d_{k+1}
    \\
      p_{k+1} \ge 0
    &\underset{\substack{k > 0 \text{ ou} \\[.2em]
                         a_0 \ge \lfloor \sqrt[n]{x} \rfloor}}{\iff}&
      a_k = \lfloor \sqrt[n]{x} \rfloor
    &\underset{\substack{k > 0 \text{ ou} \\[.2em]
                         a_0 \ge \lfloor \sqrt[n]{x} \rfloor}}{\iff}&
      a_k \le d_{k+1}
  \end{array}
\]
Dessa forma,
podemos considerar que o critério de parada é $p_{k+1} \ge 0$.
Em outras palavras,
\emph{o processo iterativo deve continuar
      apenas enquanto $a_k$ estiver diminuindo},
e quando parar de diminuir,
\emph{o valor mínimo encontrado é o próprio resultado desejado}.
Como estamos lidando apenas com valores inteiros,
isso também significa que o resultado é obtido
após um número finito de iterações.

Equivalentemente, para não ser necessário calcular o passo,
a condição de parada é $a_k \le d_{k+1}$,
o que corresponde à descrição do algoritmo,
finalizando a prova de sua corretude.
