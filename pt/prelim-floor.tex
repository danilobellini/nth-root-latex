\subsection*{Função chão}

Sejam $\gamma \in \mathds{R}$ e $k \in \mathds{Z}$.
Uma forma de definir a função chão
é como a função $\lfloor \cdot \rfloor : \mathds{R} \to \mathds{Z}$
que satisfaz:
\begin{equation}\tag{Chão}
  \lfloor \gamma \rfloor \le \gamma < \lfloor \gamma \rfloor + 1
\end{equation}
Isto é,
$\lfloor \gamma \rfloor$ é o maior inteiro menor ou igual a $\gamma$.

Somando $k$ nas inequações da definição,
o resultado também satisfaz à definição da função chão:
\[
  \overbrace{
    \underbrace{\lfloor \gamma \rfloor + k}_{\in \mathds{Z}}
  }^{\lfloor \mu \rfloor}
  \le \overbrace{\gamma + k}^{\mu} <
  \overbrace{
    \underbrace{\lfloor \gamma \rfloor + k}_{\in \mathds{Z}} + 1
  }^{\lfloor \mu \rfloor + 1}
\]
Ou seja:
\begin{equation}\label{floor-int}
  \lfloor \gamma \rfloor + k = \lfloor \gamma + k \rfloor,
  \quad \forall \gamma \in \mathds{R}, \; \forall k \in \mathds{Z}
\end{equation}

É sempre possível dizer que
$\gamma < k \implies \lfloor \gamma \rfloor < k$,
pois, pela definição dada para a função chão,
$\lfloor \gamma \rfloor \le \gamma$.
Por outro lado,
o resultado \eqref{int+1} nos diz que
quando $\lfloor \gamma \rfloor < k$ para um dado $k \in \mathds{Z}$,
temos $\lfloor \gamma \rfloor + 1 \le k$,
e podemos juntar esse resultado
com a segunda inequação da definição da função chão,
$\gamma < \lfloor \gamma \rfloor + 1$,
o que nos permite dizer que $\gamma < k$.
Ou seja, $\lfloor \gamma \rfloor < k \implies \gamma < k$.
Resumidamente:
\begin{equation}\label{floor-switch-lt}
  \gamma < k \iff \lfloor \gamma \rfloor < k,
  \quad \forall \gamma \in \mathds{R}, \; \forall k \in \mathds{Z}
\end{equation}

Uma propriedade similar pode ser encontrada
para a inequação complementar.
Como $\lfloor \gamma \rfloor \le \gamma$,
temos $k \le \lfloor \gamma \rfloor \implies k \le \gamma$.
Por outro lado,
como $\gamma < \lfloor \gamma \rfloor + 1$,
se admitirmos $k \le \gamma$,
temos $k < \lfloor \gamma \rfloor + 1$,
e por esta ser uma inequação de inteiros,
o resultado \eqref{int+1} nos permite dizer que
$k < \lfloor \gamma \rfloor + 1 \iff
 k + 1 \le \lfloor \gamma \rfloor + 1 \iff
 k \le \lfloor \gamma \rfloor$.
Portanto:
\begin{equation}\label{floor-switch-ge}
  \gamma \ge k \iff \lfloor \gamma \rfloor \ge k,
  \quad \forall \gamma \in \mathds{R}, \; \forall k \in \mathds{Z}
\end{equation}

A última justificativa fornecida
também se aplica quando admitimos $k < \gamma$,
de forma que:
\begin{equation}\label{floor-lt2ge-int}
  \gamma > k
  \implies
  \lfloor \gamma \rfloor \ge k
  \quad \forall \gamma \in \mathds{R}, \; \forall k \in \mathds{Z}
\end{equation}

Aplicando a definição da função chão
sobre $\lfloor \gamma \rfloor / n$, $n \in \mathds{N}^*$,
e multiplicando o resultado por $n$, temos:
\[
      \left\lfloor \dfrac{\lfloor \gamma \rfloor}{n} \right\rfloor
    \le
      \dfrac{\left\lfloor \gamma \right\rfloor}{n}
    <
      \left\lfloor \dfrac{\lfloor \gamma \rfloor}{n} \right\rfloor
      + 1
  \quad\iff\quad
      n \left\lfloor \dfrac{\lfloor \gamma \rfloor}{n} \right\rfloor
    \le
      \left\lfloor \gamma \right\rfloor
    <
      n \left\lfloor \dfrac{\lfloor \gamma \rfloor}{n} \right\rfloor
      + n
  \quad\stackfirst{
    m =
    \left\lfloor
      {}^{\scriptstyle\lfloor \gamma \rfloor} / {}_{\scriptstyle n}
    \right\rfloor
  }{\iff}\quad
    n m \le \left\lfloor \gamma \right\rfloor < n m + n
\]

O resultado \eqref{floor-switch-lt}
aplicado à desigualdade $\left\lfloor \gamma \right\rfloor < n m + n$
resulta em $\gamma < n m + n$,
e o resultado \eqref{floor-switch-ge} nos diz que
a desigualdade $n m \le \left\lfloor \gamma \right\rfloor$
pode ser escrita como $n m \le \gamma$.
Continuando:
\[
  n m \le \left\lfloor \gamma \right\rfloor < n m + n
  \quad\stackfirst{\eqref{floor-switch-lt}\text{ e }
                   \eqref{floor-switch-ge}}{\iff}\quad
  n m \le \gamma < n m + n
  \quad\stackfirst{n > 0}{\iff}\quad
  m \le \dfrac{\gamma}{n} < m + 1
\]
Que consiste novamente na definição da função chão, pois $m$ é inteiro.
Isto é, $m = \lfloor \gamma / n \rfloor$.
Finalmente:
\begin{equation}\label{floor-nested-ratio}
  \left\lfloor \dfrac{\lfloor \gamma \rfloor}{n} \right\rfloor
  =
  \left\lfloor \dfrac{\gamma\vphantom{l}}{n} \right\rfloor
  \quad \forall \gamma \in \mathds{R}, \; \forall n \in \mathds{N}^*
\end{equation}
