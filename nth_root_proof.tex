\documentclass{article}

\usepackage{mainstyle}

\title{Algoritmo para calcular $\lfloor \sqrt[n]{x} \rfloor$
       numericamente com $x, n \in \mathds{N}^*$}

\author{Danilo J. S. Bellini}


\begin{document}

\maketitle

\section*{Resumo}

Na PEP 572, há um algoritmo\footnote{
  \url{https://www.python.org/dev/peps/pep-0572/\#a-numeric-example}
}
implementado no Python com e sem uma \emph{assignment expression}
para o cálculo da raiz $n$-ésima de $x$,
utilizando apenas inteiros (entrada, saída e valores intermediários),
sendo que o resultado final é o valor truncado da raiz desejada
no caso deste não ser inteiro.
O objetivo deste texto é explicar, justificar, provar e analisar
esse algoritmo e seu funcionamento.

\section*{Introdução}

\subsection*{Descrição do algoritmo}

Sejam $x$ e $n$ valores inteiros positivos,
e $a_0 \ge \lfloor \sqrt[n]{x} \rfloor$ um palpite inicial\footnote{
  Na PEP 572,
  a restrição para o palpite inicial é $a_0 \ge \sqrt[n]{x}$.
  A diferença é sutil, mas é possível notar,
  com base em \eqref{ak-between-floor-and-root},
  que o único número inteiro que satisfaz apenas uma das desigualdades
  é o próprio resultado desejado para o algoritmo,
  situação na qual se espera que o processo iterativo encerre
  e o próprio valor do palpite inicial $a_0$ seja devolvido.
},
também inteiro e positivo.
Para encontrar o valor de $\lfloor \sqrt[n]{x} \rfloor$,
calcularemos iterativamente a seguinte sequência
enquanto $a_k > d_{k+1}$:
\[
  d_{k+1} = \left\lfloor \dfrac{x}{a_k^{n-1}} \right\rfloor
\]
\[
  a_{k+1} = \left\lfloor \dfrac{(n-1) a_k + d_{k+1}}{n} \right\rfloor
\]
O resultado é o último valor $a_k$ encontrado nesse processo,
isto é, o resultado é o valor de $a_r$,
em que $r$ é o menor índice $k$ para o qual vale $a_k \le d_{k+1}$.

\subsection*{Implementação do algoritmo em Python}

O algoritmo proposto pode ser escrito em Python,
utilizando uma \emph{assignment expression}\footnote{
  Expressão realizando uma atribuição.
  O uso dessa sintaxe com o operador ``\texttt{:=}'' (\emph{walrus})
  está disponível a partir do Python 3.8.
}:

\begin{center}
  \begin{minipage}{7cm}
    \inputminted{python}{nth_root.py}
  \end{minipage}
\end{center}

A menos de espaços em branco,
as $3$ últimas linhas desse algoritmo
formam exatamente o fragmento de código que consta na PEP 572.
A primeira linha apenas define um bloco como uma \emph{função},
no sentido em essa palavra é usada
no contexto de linguagens de programação.
A segunda linha define um valor inicial específico,
uma novidade do presente texto, a qual não consta na PEP 572.


\section*{Considerações preliminares}

Um resultado relevante acerca da comparação de números inteiros
consiste em usar o ``passo'' unitário
para converter entre desigualdades estritas e não-estritas.
Dado $m \in \mathds{Z}$, sabemos que $m > 0 \iff m \ge 1$,
isto é, $1$ é o ínfimo do conjunto dos inteiros positivos.
Escrevendo $m$ como uma subtração de inteiros $m = \ell - k$, temos:
\begin{equation}\label{int+1}
  k < \ell \iff k + 1 \le \ell,
  \quad \forall k \in \mathds{Z}, \; \forall \ell \in \mathds{Z}
\end{equation}

\subsection*{Função chão}

Sejam $\gamma \in \mathds{R}$ e $k \in \mathds{Z}$.
Uma forma de definir a função chão
é como a função $\lfloor \cdot \rfloor : \mathds{R} \to \mathds{Z}$
que satisfaz:
\begin{equation}\tag{Chão}
  \lfloor \gamma \rfloor \le \gamma < \lfloor \gamma \rfloor + 1
\end{equation}
Isto é,
$\lfloor \gamma \rfloor$ é o maior inteiro menor ou igual a $\gamma$.

Somando $k$ nas inequações da definição,
o resultado também satisfaz à definição da função chão:
\[
  \overbrace{
    \underbrace{\lfloor \gamma \rfloor + k}_{\in \mathds{Z}}
  }^{\lfloor \mu \rfloor}
  \le \overbrace{\gamma + k}^{\mu} <
  \overbrace{
    \underbrace{\lfloor \gamma \rfloor + k}_{\in \mathds{Z}} + 1
  }^{\lfloor \mu \rfloor + 1}
\]
Ou seja:
\begin{equation}\label{floor-int}
  \lfloor \gamma \rfloor + k = \lfloor \gamma + k \rfloor,
  \quad \forall \gamma \in \mathds{R}, \; \forall k \in \mathds{Z}
\end{equation}

É sempre possível dizer que
$\gamma < k \implies \lfloor \gamma \rfloor < k$,
pois, pela definição dada para a função chão,
$\lfloor \gamma \rfloor \le \gamma$.
Por outro lado,
o resultado \eqref{int+1} nos diz que
quando $\lfloor \gamma \rfloor < k$ para um dado $k \in \mathds{Z}$,
temos $\lfloor \gamma \rfloor + 1 \le k$,
e podemos juntar esse resultado
com a segunda inequação da definição da função chão,
$\gamma < \lfloor \gamma \rfloor + 1$,
o que nos permite dizer que $\gamma < k$.
Ou seja, $\lfloor \gamma \rfloor < k \implies \gamma < k$.
Resumidamente:
\begin{equation}\label{floor-switch-lt}
  \gamma < k \iff \lfloor \gamma \rfloor < k,
  \quad \forall \gamma \in \mathds{R}, \; \forall k \in \mathds{Z}
\end{equation}

Uma propriedade similar pode ser encontrada
para a inequação complementar.
Como $\lfloor \gamma \rfloor \le \gamma$,
temos $k \le \lfloor \gamma \rfloor \implies k \le \gamma$.
Por outro lado,
como $\gamma < \lfloor \gamma \rfloor + 1$,
se admitirmos $k \le \gamma$,
temos $k < \lfloor \gamma \rfloor + 1$,
e por esta ser uma inequação de inteiros,
o resultado \eqref{int+1} nos permite dizer que
$k < \lfloor \gamma \rfloor + 1 \iff
 k + 1 \le \lfloor \gamma \rfloor + 1 \iff
 k \le \lfloor \gamma \rfloor$.
Portanto:
\begin{equation}\label{floor-switch-ge}
  \gamma \ge k \iff \lfloor \gamma \rfloor \ge k,
  \quad \forall \gamma \in \mathds{R}, \; \forall k \in \mathds{Z}
\end{equation}

A última justificativa fornecida
também se aplica quando admitimos $k < \gamma$,
de forma que:
\begin{equation}\label{floor-lt2ge-int}
  \gamma > k
  \implies
  \lfloor \gamma \rfloor \ge k
  \quad \forall \gamma \in \mathds{R}, \; \forall k \in \mathds{Z}
\end{equation}

Aplicando a definição da função chão
sobre $\lfloor \gamma \rfloor / n$, $n \in \mathds{N}^*$,
e multiplicando o resultado por $n$, temos:
\[
      \left\lfloor \dfrac{\lfloor \gamma \rfloor}{n} \right\rfloor
    \le
      \dfrac{\left\lfloor \gamma \right\rfloor}{n}
    <
      \left\lfloor \dfrac{\lfloor \gamma \rfloor}{n} \right\rfloor
      + 1
  \quad\iff\quad
      n \left\lfloor \dfrac{\lfloor \gamma \rfloor}{n} \right\rfloor
    \le
      \left\lfloor \gamma \right\rfloor
    <
      n \left\lfloor \dfrac{\lfloor \gamma \rfloor}{n} \right\rfloor
      + n
  \quad\stackfirst{
    m =
    \left\lfloor
      {}^{\scriptstyle\lfloor \gamma \rfloor} / {}_{\scriptstyle n}
    \right\rfloor
  }{\iff}\quad
    n m \le \left\lfloor \gamma \right\rfloor < n m + n
\]

O resultado \eqref{floor-switch-lt}
aplicado à desigualdade $\left\lfloor \gamma \right\rfloor < n m + n$
resulta em $\gamma < n m + n$,
e o resultado \eqref{floor-switch-ge} nos diz que
a desigualdade $n m \le \left\lfloor \gamma \right\rfloor$
pode ser escrita como $n m \le \gamma$.
Continuando:
\[
  n m \le \left\lfloor \gamma \right\rfloor < n m + n
  \quad\stackfirst{\eqref{floor-switch-lt}\text{ e }
                   \eqref{floor-switch-ge}}{\iff}\quad
  n m \le \gamma < n m + n
  \quad\stackfirst{n > 0}{\iff}\quad
  m \le \dfrac{\gamma}{n} < m + 1
\]
Que consiste novamente na definição da função chão, pois $m$ é inteiro.
Isto é, $m = \lfloor \gamma / n \rfloor$.
Finalmente:
\begin{equation}\label{floor-nested-ratio}
  \left\lfloor \dfrac{\lfloor \gamma \rfloor}{n} \right\rfloor
  =
  \left\lfloor \dfrac{\gamma\vphantom{l}}{n} \right\rfloor
  \quad \forall \gamma \in \mathds{R}, \; \forall n \in \mathds{N}^*
\end{equation}

\subsection*{Função teto e sua relação com a função chão}

A função teto pode ser definida
como a função $\lceil \cdot \rceil : \mathds{R} \to \mathds{Z}$
que satisfaz:
\begin{equation}\tag{Teto}
  \lceil \gamma \rceil - 1 < \gamma \le \lceil \gamma \rceil
\end{equation}
Isto é,
$\lceil \gamma \rceil$ é o menor inteiro maior ou igual a $\gamma$.

Pelas definições das funções chão e teto,
quando $\gamma$ é inteiro, temos:
\begin{equation}\label{ceil-floor-z}
  \lfloor \gamma \rfloor = \gamma = \lceil \gamma \rceil
  \quad \forall \gamma \in \mathds{Z}
\end{equation}
E, para o cenário complementar, quando $\gamma \notin \mathds{Z}$,
podemos explicitar que as inequações das definições não são estritas,
isto é, $\lfloor \gamma \rfloor < \gamma < \lfloor \gamma \rfloor + 1$
e $\lceil \gamma \rceil - 1 < \gamma < \lceil \gamma \rceil$.
Pela propriedade transitiva, podemos obter duas novas inequações:
$\lfloor \gamma \rfloor < \lceil \gamma \rceil$
e $\lceil \gamma \rceil - 1 < \lfloor \gamma \rfloor + 1$.
O resultado \eqref{int+1}
aplicado a cada uma dessas duas inequações nos fornece
$\lfloor \gamma \rfloor + 1 \le \lceil \gamma \rceil$ e
$\lceil \gamma \rceil \le \lfloor \gamma \rfloor + 1$, respectivamente.
Isso justifica a igualdade
$\lceil \gamma \rceil = \lfloor \gamma \rfloor + 1$,
que pode ser reescrita como:
\begin{equation}\label{ceil-floor-notz}
  \lceil \gamma \rceil - \lfloor \gamma \rfloor = 1
  \quad \forall \gamma \in \mathds{R} \setminus \mathds{Z}
\end{equation}
Podemos aplicar esses últimos resultados
\eqref{ceil-floor-z} e \eqref{ceil-floor-notz}
em meio a provas de expressões
que envolvam simultaneamente as funções chão e teto,
bastando utilizar o particionamento entre inteiros e não-inteiros
para provar cada caso separadamente.
Esse recurso será usado no que segue.

Sejam $x \in \mathds{Z}$ e $n \in \mathds{N}^*$.
Quando $x/n \in \mathds{Z}$, podemos escrever:
\[
    \left\lfloor \dfrac{x - 1}{n} \right\rfloor =
    \left\lfloor \dfrac{x}{n} - \dfrac{1}{n} \right\rfloor
      \stackfirst{\text{\eqref{floor-int}}}{=}
    \dfrac{x}{n} + \left\lfloor - \dfrac{1}{n} \right\rfloor =
    \dfrac{x}{n} - 1
  \quad\implies\quad
    \left\lfloor \dfrac{x - 1}{n} \right\rfloor + 1 =
    \dfrac{x}{n}
      \stackfirst{(\mathds{Z})}{=}
    \left\lceil \dfrac{x\vphantom{l}}{n} \right\rceil
\]
Pela aplicação da propriedade
descrita na equação \eqref{floor-int}
e pelo fato de que $\lfloor -1/n \rfloor = -1$,
visto que $-1 \le -1/n < 0$.
Por outro lado, quando $x/n \notin \mathds{Z}$,
partindo da definição da função chão,
já descartando a igualdade por ela ser exclusiva dos inteiros,
e utilizando os resultados \eqref{int+1} e \eqref{floor-switch-ge},
podemos escrever:
\begin{align*}
    \left\lfloor \dfrac{x\vphantom{l}}{n} \right\rfloor
    < \dfrac{x\vphantom{l}}{n}
  &\quad\stackrel{\rule[-.3em]{0pt}{0pt}n > 0}{\iff}\quad
    n \left\lfloor \dfrac{x\vphantom{l}}{n} \right\rfloor < x
  \\&\quad\iff\quad
    0 < x - n \left\lfloor \dfrac{x\vphantom{l}}{n} \right\rfloor
  \\&\quad\stackrel{\rule[-.3em]{0pt}{0pt}\eqref{int+1}}{\iff}\quad
    1 \le x - n \left\lfloor \dfrac{x\vphantom{l}}{n} \right\rfloor
  \\&\quad\iff\quad
    n \left\lfloor \dfrac{x\vphantom{l}}{n} \right\rfloor \le x - 1
  \\&\quad\stackrel{\rule[-.3em]{0pt}{0pt}n > 0}{\iff}\quad
    \left\lfloor \dfrac{x\vphantom{l}}{n} \right\rfloor
    \le \dfrac{x - 1}{n}
  \\&\quad\stackrel{\rule[-.3em]{0pt}{0pt}\eqref{floor-switch-ge}}{\iff}\quad
    \left\lfloor \dfrac{x\vphantom{l}}{n} \right\rfloor
    \le \left\lfloor \dfrac{x - 1}{n} \right\rfloor
\end{align*}
Podemos também aproveitar o resultado \eqref{floor-lt2ge-int}
e a definição da função chão
$\left\lfloor \dfrac{x - 1}{n} \right\rfloor \le \dfrac{x - 1}{n}$ em:
\[
    -1 < 0
  \quad\iff\quad
    x - 1 < x
  \quad\stackfirst{n > 0}{\iff}\quad
    \dfrac{x - 1}{n} < \dfrac{x}{n}
  \quad\stackfirst{\text{(Chão)}}{\implies}\quad
    \left\lfloor \dfrac{x - 1}{n} \right\rfloor < \dfrac{x}{n}
  \quad\stackfirst{\eqref{floor-lt2ge-int}}{\iff}\quad
    \left\lfloor \dfrac{x - 1}{n} \right\rfloor
    \le \left\lfloor \dfrac{x\vphantom{l}}{n} \right\rfloor
\]
Ou seja,
$\left\lfloor \dfrac{x - 1}{n} \right\rfloor =
 \left\lfloor \dfrac{x\vphantom{l}}{n} \right\rfloor$
quando $x/n \notin \mathds{Z}$.
Mas nesse caso também sabemos que
$\left\lceil \dfrac{x\vphantom{l}}{n} \right\rceil -
 \left\lfloor \dfrac{x\vphantom{l}}{n} \right\rfloor = 1$.
Logo:
\[
  \left\lceil \dfrac{x\vphantom{l}}{n} \right\rceil
  = 1 + \left\lfloor \dfrac{x\vphantom{l}}{n} \right\rfloor
  = 1 + \left\lfloor \dfrac{x - 1}{n} \right\rfloor
\]
Que é o mesmo resultado encontrado quando $x/n \in \mathds{Z}$.
Com isso, concluímos que:
\begin{equation}\label{ceil-x-over-n-as-floor}
  \left\lceil \dfrac{x\vphantom{l}}{n} \right\rceil
  = \left\lfloor \dfrac{x - 1}{n} \right\rfloor + 1
  \quad \forall x \in \mathds{Z}, \; \forall n \in \mathds{N}^*
\end{equation}
Esse valor é precisamente a equação da condição inicial
adotada para o algoritmo.

\subsection*{Desigualdade entre as médias aritmética e geométrica}

Dados $n \in \mathds{N}^*$ e números reais $y_i \ge 0$,
a desigualdade entre as médias aritmética e geométrica
consiste em dizer que
\emph{a média aritmética é maior ou igual à média geométrica}, isto é:
\[\tag{MA $\ge$ MG}
    \overbrace{
      \dfrac{1}{n} \sum_{i=1}^n y_i
    }^{\mathclap{\text{Média aritmética}}}
  \ge
    \underbrace{
      \sqrt[n]{\prod_{i=1}^n y_i}
    }_{\mathclap{\text{Média geométrica}}}
\]
Para $n = 1$ esse resultado é imediato,
visto que ambas as médias de um único elemento
são iguais a esse próprio elemento.
Para $n = 2$, também sabemos que a desigualdade é verdadeira, pois:
\begin{align*}
    (y_1 - y_2)^2 \ge 0
  &\iff
    y_1^2 - 2 y_1 y_2 + y_2^2 \ge 0
  \\&\iff
    y_1^2 + 2 y_1 y_2 + y_2^2 \ge 4 y_1 y_2
  \\&\iff
    (y_1 + y_2)^2 \ge 4 y_1 y_2
  \\&\stackrel{\rule[-.3em]{0pt}{0pt}y_i \ge 0}{\iff}\vphantom{\int}
    y_1 + y_2 \ge 2 \sqrt{y_1 y_2}
  \\&\iff
    \dfrac{y_1 + y_2}{2} \ge \sqrt{y_1 y_2}
  \tag{B}
\end{align*}

É possível provar essa desigualdade para qualquer $n \in \mathds{N}^*$
por meio de duas induções em direções opostas.
Para a primeira indução, admita que $n = 1$ ou $n = 2$ é o caso base.
Se a desigualdade vale para $n$ elementos,
ela também valerá para $2n$ elementos, pois:
\[
    \dfrac{1}{2n} \sum_{i=1}^{2n} y_i
  =
    \dfrac{1}{2}
    \left(
      \dfrac{1}{n} \sum_{i=1}^{n} y_i +
      \dfrac{1}{n} \sum_{i=n+1}^{2n} y_i
    \right)
  \stackfirst{\text{(H)}}{\ge}
    \dfrac{1}{2}
    \left(\vphantom{\sqrt{\prod_{i=1}^n}}\right.
      \underbrace{\sqrt[n]{\prod_{i=1}^n y_i}}_{z_1} +
      \underbrace{\sqrt[n]{\prod_{i=n+1}^{2n} y_i}}_{z_2}
    \left.\vphantom{\sqrt{\prod_{i=1}^n}}\right)
  \stackfirst{\text{(B)}}{\ge}
    \underbrace{
      \sqrt{ \left( \sqrt[n]{\prod_{i=1}^n y_i} \right)
             \left( \sqrt[n]{\prod_{i=n+1}^{2n} y_i} \right) }
    }_{\sqrt{z_1 z_2}}
\]\[
    \dfrac{1}{2n} \sum_{i=1}^{2n} y_i
  \ge
    \sqrt{ \sqrt[n]{ \left( \prod_{i=1}^n y_i \right)
                     \left( \prod_{i=n+1}^{2n} y_i \right) } }
  =
    \sqrt[2n]{\prod_{i=1}^{2n} y_i}
\]
Em que (H) denota o uso da \emph{hipótese} dessa indução
para os $2$ conjuntos,
e (B) denota o uso do resultado encontrado para $n = 2$
aplicado aos valores $z_i$ indicados.
Isso prova que a desigualdade entre as médias aritmética e geométrica
vale para todas as potências de $2$, isto é,
para $n = 2^m$, $m \in \mathds{N}^*$.
Com isso já sabemos que não há um valor máximo
para o qual a desigualdade vale\footnote{
  Se adotarmos $w = \lceil\log_2 n\rceil$, sabemos que $n \le 2^w$
  pelo fato de todas as funções envolvidas serem não-decrescentes.
  Para qualquer $n$,
  podemos construir uma potência de $2$ maior ou igual a $n$,
  e $2^w$ é a menor potência de $2$ que satisfaz esse requisito.
},
e podemos realizar uma indução no sentido inverso,
na qual podemos adotar qualquer potência de $2$ como novo caso base.

Adotando $n = 2^m$ para qualquer $m \in \mathds{N}^*$ como caso base,
e admitindo como hipótese de indução
que a desigualdade entre as médias aritmética e geométrica
vale para $n$ elementos,
queremos mostrar que ela também valerá para $n-1$ elementos.
Para isso, podemos considerar que temos apenas $n-1$ elementos $y_i$,
e inserimos o $n$-ésimo elemento como a média aritmética dos demais.
Isto é, considere:
\[
  y_n = \dfrac{1}{n-1} \sum_{i=1}^{n-1} y_i
\]
Isso já nos obriga a utilizar $n \ge 2$
como restrição deste processo,
para evitar uma divisão por zero.
Utilizando esse $n$-ésimo elemento, a média aritmética torna-se:
\[
    \dfrac{1}{n} \sum_{i=1}^{n} y_i
  =
    \dfrac{1}{n} \left( y_n + \sum_{i=1}^{n-1} y_i \right)
  =
    \dfrac{y_n}{n} + \dfrac{1}{n} \sum_{i=1}^{n-1} y_i
  =
    \dfrac{y_n}{n} + \dfrac{(n - 1) y_n}{n}
  =
    y_n
  \stackfirst{\text{(H)}}{\ge}
    \sqrt[n]{\prod_{i=1}^n y_i}
  =
    \sqrt[n]{y_n \prod_{i=1}^{n-1} y_i}
\]
Em que o (H) denota o uso da hipótese dessa indução
com base na expressão mais à esquerda,
utilizando o fato de que os passos intermediários são todos iguais.
Caso $y_n > 0$, temos:
\[
    y_n \ge \sqrt[n]{y_n \prod_{i=1}^{n-1} y_i}
  \;\implies\;
    y_n^n \ge y_n \prod_{i=1}^{n-1} y_i
  \;\stackfirst{y_n > 0}{\implies}\;
    y_n^{n-1} \ge \prod_{i=1}^{n-1} y_i
  \;\implies\;
    y_n \ge \sqrt[n-1]{\prod_{i=1}^{n-1} y_i}
\]
Que é exatamente a desigualdade entre as médias aritmética e geométrica
para $n - 1$ elementos.
Caso $y_n = 0$, todos os elementos $y_i$ precisarão ser iguais a zero,
visto que esta é uma soma de números não-negativos,
e isso resulta na igualdade das duas médias (ambas iguais a zero),
o que também satisfaz
a desigualdade entre as médias aritmética e geométrica.

Dado que o caso base pode ser arbitrariamente grande,
essa segunda indução mostra que a desigualdade é válida
para qualquer $n \ge 2$,
``preenchendo as lacunas'' dos números que não são potências de $2$.

\subsection*{Convergência de sequências}

Uma \emph{sequência} de números reais
é uma função $q:\mathds{N}\to\mathds{R}$
para a qual denotamos como $q_k$ seu $k$-ésimo elemento,
também chamado de \emph{termo geral}.
Podemos construir uma sequência por meio de uma \emph{recorrência},
que consiste em definir cada novo elemento da sequência
como uma função dos elementos anteriores,
desde que fixados valores iniciais
que permitam construir o restante da sequência.
Em particular, o algoritmo proposto
está escrito na forma de uma recorrência.

Uma sequência é dita:
\begin{itemize}
  \item
    \emph{Limitada} se existe $M \ge 0$ real
    tal que $|q_k| \le M$ para todo $k$.
    Ou, equivalentemente, se existem $M_{\min}$ e $M_{\max}$
    tais que  $M_{\min} \le q_k \le M_{\max}$ para todo $k$.
  \item
    \emph{Crescente} ou \emph{estritamente crescente}
    quando $q_{k+1} > q_k$.
  \item
    \emph{Não-crescente} quando $q_{k+1} \le q_k$.
  \item
    \emph{Decrescente} ou \emph{estritamente decrescente}
    quando $q_{k+1} < q_k$.
  \item
    \emph{Não-decrescente} quando $q_{k+1} \ge q_k$.
  \item
    \emph{Monotônica} ou \emph{monótona} se, para todo $k$,
    ela for ou não-crescente, ou não-decrescente.
    Isso também inclui os casos
    em que a sequência é ou estritamente crescente,
    ou estritamente decrescente.
  \item
    \emph{Convergente} quando
    existe um \emph{limite} $L$ para a sequência,
    isto é, quando,
    para qualquer $\varepsilon > 0$,
    existe um $K$ tal que $|q_k - L| < \varepsilon$
    para todo $k \ge K$.
    Tal limite pode ser indicado como
    ``$q_k \to L$ quando $k \to \infty$'',
    ou $\displaystyle \lim_{k\to\infty} q_k = L$.
  \item
    \emph{Divergente} quando não é convergente.
\end{itemize}

\textbf{Se a sequência é convergente, seu limite é único}.
Para provar, suponha que existam dois limites distintos, $L_1$ e $L_2$,
e, sem perda de generalidade, admita $L_2 > L_1$.
Isso se revela uma contradição,
pois adotando qualquer valor de $\varepsilon$
que satisfaça $0 < \varepsilon \le (L_2 - L_1)/2$,
podemos justificar, por contraposição,
que pelo menos um dos limites é falso:
\[
  \begin{array}{l}
        |q_k - L_1| < \varepsilon
      \iff
        - \varepsilon < q_k - L_1 < \varepsilon
      \iff
        L_1 - \varepsilon <
        \overbrace{q_k < L_1 + \varepsilon}^{(\varepsilon_1)}
    \\
        |q_k - L_2| < \varepsilon
      \iff
        - \varepsilon < q_k - L_2 < \varepsilon
      \iff
        \underbrace{L_2 - \varepsilon < q_k}_{(\varepsilon_2)}
        < L_2 + \varepsilon
  \end{array}
  \Bigg\}
  \stackfirst{\mathclap{
    (\varepsilon_1) \text{ e } (\varepsilon_2)
  }}{\implies}
    L_2 - \varepsilon < L_1 + \varepsilon
  \iff
    \varepsilon > \dfrac{L_2 - L_1}{2}
\]

\textbf{Se a sequência é convergente, então ela é limitada}.
Como a sequência é convergente,
há um limite $L$ para o qual ela converge.
Fixe um valor real positivo para o $\varepsilon$.
A definição do limite nos diz que existe um $K$
a partir do qual vale $|q_k - L| < \varepsilon$,
o que nos permite particionar a sequência em duas:
os $K$ elementos iniciais da sequência,
e os elementos a partir dos quais
sempre vale $|q_k - L| < \varepsilon$.
Tome
$\displaystyle
 M = \max\left(|L| + \varepsilon, \max_{0 \le k < K} |q_k|\right)$,
então $|q_k| \le M$ para todo $k$,
pois o conjunto dos $K$ elementos iniciais da sequência é limitado
(i.e., possui valores máximo e mínimo)
e os elementos seguintes podem ser majorados por $|L| + \varepsilon$,
que é um número positivo:
\[
  \begin{array}{c}\displaystyle
      |q_k - L| < \varepsilon
    \iff
      - \varepsilon < q_k - L < \varepsilon
    \iff
      \tikzeq{seqlimit-above}{
        L - \varepsilon < q_k < L + \varepsilon
      }
    \\[1em] \displaystyle
      -|L| - \varepsilon \le
        \tikzeq{seqlimit-below}{
          L - \varepsilon < q_k < L + \varepsilon
        }
      \le |L| + \varepsilon
    \implies
      -(|L| + \varepsilon) < q_k < |L| + \varepsilon
    \iff
      q_k < |L| + \varepsilon
    \begin{tikzpicture}[eq-overlay]
      \draw[brace80]
        (seqlimit-above.south east) -- (seqlimit-above.south west);
      \draw[brace80]
        (seqlimit-below.north west) -- (seqlimit-below.north east);
      \coordinate (seqlimit-below-tip) at
        ($(seqlimit-below.north west)!.8!(seqlimit-below.north east)
          + (0, .5em)$);
      \coordinate (seqlimit-above-tip) at
        ($(seqlimit-above.south east)!.8!(seqlimit-above.south west)
          + (0, -.5em)$);
      \draw[out=210, in=30, distance=2em, ->,
            arrows=-{Latex[length=2mm, width=1mm]}]
        (seqlimit-above-tip) to (seqlimit-below-tip);
    \end{tikzpicture}
  \end{array}
\]

\textbf{Se a sequência é monotônica e limitada,
        então ela é convergente}.
Dizer que a sequência é limitada
corresponde a dizer que o conjunto de todos os elementos da sequência
é limitado tanto superiormente como inferiormente.
A completude dos números reais\footnote{
  O axioma da completude dos números reais
  também é conhecido como axioma do supremo,
  propriedade do menor limitante superior,
  ou cota superior.
  Esse axioma diz que
  todo conjunto de números reais que é limitado superiormente
  possui um número real como menor limitante superior,
  chamado de \emph{supremo},
  mesmo que este não pertença ao próprio conjunto.
  Isto significa dizer que,
  dado um conjunto $B \subset \mathds{R}$ limitado superiormente,
  para todo $\varepsilon > 0$
  existe um $b \in B$
  tal que $b > (\sup B) - \varepsilon$.
  Analogamente
  para conjuntos de números reais limitados inferiormente,
  o \emph{ínfimo} é o maior limitante inferior do conjunto,
  como consequência da aplicação desse axioma
  sobre $C = \{-b: b \in B\}$,
  isto é,
  dado um conjunto $C \subset \mathds{R}$ limitado inferiormente,
  para todo $\varepsilon > 0$
  existe um $c \in C$
  tal que $c < (\inf C) + \varepsilon$.
} garante que, nesse caso, o conjunto de valores reais $q_k$
possui um supremo e um ínfimo.
Caso seja não-decrescente, a sequência converge para o supremo
$S = \sup \{q_k\}$,
e caso seja não-crescente, a sequência converge para o ínfimo
$I = \inf \{q_k\}$,
pois:
\[
  \begin{array}{r|rcl|rcl|c}
      \multicolumn{1}{r}{} &  % Remove border
      \multicolumn{3}{c}{\textbf{Não-decrescente}} &
      \multicolumn{3}{c}{\textbf{Não-crescente}}
    \\ \cline{2-7}
      \text{Completude (axioma)}
      &
      \forall \varepsilon \exists k_s :
        \varepsilon > 0 &\implies& q_{k_s} > S - \varepsilon
      &
      \forall \varepsilon \exists k_i :
        \varepsilon > 0 &\implies& q_{k_i} < I + \varepsilon
      & \text{(A)}
    \\
      \text{Limitada pelo supremo/ínfimo}
      &
      && S \ge q_k
      &
      && I \le q_k
      & \text{(L)}
    \\
      \text{Monotônica}
      &
      k > k_s &\implies& q_k \ge q_{k_s}
      &
      k > k_i &\implies& q_k \le q_{k_i}
      & \text{(M)}
    \\ \cline{2-7}
  \end{array}
\]
\[
  \begin{array}{r@{\qquad}l}
      \hfill\textbf{Monotônica não-decrescente e limitada}\hfill
    &
      \hfill\textbf{Monotônica não-crescente e limitada}\hfill
    \\[1.5em]
      \underbracket{
        \forall \varepsilon \exists k_s :
        \substack{
          (\varepsilon > 0) \\[.3em]
          \text{e} \\[.3em]
          (k > k_s)
        }
        {\displaystyle\Bigg\}}
        {\Rightarrow}
        \left\{
          \begin{array}{c}
              \tikzeq{sup-a}{S - \varepsilon}
                <
              \tikzeq{sup-b}{q_{k_s}}
                \le
              \tikzeq{sup-c}{q_k}
                \le
              \tikzeq{sup-d}{S}
                <
              \tikzeq{sup-e}{S + \varepsilon}
              \begin{tikzpicture}[eq-overlay]
                \draw[brace]
                  ($(sup-a.base west) + (0, .7em)$)
                  -- node[above=.2em] {(A)}
                  ($(sup-b.base east) + (0, .7em)$);
                \draw[brace]
                  ($(sup-c.base west) + (0, .7em)$)
                  -- node[above=.2em] {(L)}
                  ($(sup-d.base east) + (0, .7em)$);
                \draw[brace80]
                  ($(sup-c.base east) + (0, -.2em)$)
                  -- node[below=.3em, pos=.8] {(M)}
                  ($(sup-b.base west) + (0, -.2em)$);
                \draw[brace80, decoration={mirror}]
                  ($(sup-d.base west) + (0, -.1em)$)
                  -- node[below=.3em, pos=.8] {$\varepsilon > 0$}
                  ($(sup-e.base east) + (0, -.1em)$);
              \end{tikzpicture}
            \\[.6em]
              \big\Downarrow
            \\
              S - \varepsilon < q_k < S + \varepsilon
            \\
              \Updownarrow
            \\
              - \varepsilon < q_k - S < \varepsilon
            \\
              \Updownarrow
            \\
              |q_k - S| < \varepsilon
          \end{array}
        \right.
      }_{\substack{
          \displaystyle \Downarrow
        \\[.3em]
          \displaystyle \lim_{k\to\infty} q_k = S = \sup \{q_k\}
      }}
    &
      \underbracket{
        \forall \varepsilon \exists k_i :
        \substack{
          (\varepsilon > 0) \\[.3em]
          \text{e} \\[.3em]
          (k > k_i)
        }
        {\displaystyle\Bigg\}}
        {\Rightarrow}
        \left\{
          \begin{array}{c}
              \tikzeq{inf-a}{I - \varepsilon}
                <
              \tikzeq{inf-b}{I}
                \le
              \tikzeq{inf-c}{q_k}
                \le
              \tikzeq{inf-d}{q_{k_i}}
                <
              \tikzeq{inf-e}{I + \varepsilon}
              \begin{tikzpicture}[eq-overlay]
                \draw[brace, decoration={mirror}]
                  ($(inf-e.base east) + (0, .7em)$)
                  -- node[above=.2em] {(A)}
                  ($(inf-d.base west) + (0, .7em)$);
                \draw[brace, decoration={mirror}]
                  ($(inf-c.base east) + (0, .7em)$)
                  -- node[above=.2em] {(L)}
                  ($(inf-b.base west) + (0, .7em)$);
                \draw[brace80, decoration={mirror}]
                  ($(inf-c.base west) + (0, -.2em)$)
                  -- node[below=.3em, pos=.8] {(M)}
                  ($(inf-d.base east) + (0, -.2em)$);
                \draw[brace80]
                  ($(inf-b.base east) + (0, -.1em)$)
                  -- node[below=.3em, pos=.8] {$- \varepsilon < 0$}
                  ($(inf-a.base west) + (0, -.1em)$);
              \end{tikzpicture}
            \\[.6em]
              \big\Downarrow
            \\
              I - \varepsilon < q_k < I + \varepsilon
            \\
              \Updownarrow
            \\
              - \varepsilon < q_k - I < \varepsilon
            \\
              \Updownarrow
            \\
              |q_k - I| < \varepsilon
          \end{array}
        \right.
      }_{\substack{
          \displaystyle \Downarrow
        \\[.3em]
          \displaystyle \lim_{k\to\infty} q_k = I = \inf \{q_k\}
      }}
  \end{array}
\]

\textbf{Se a sequência converge,
        a diferença entre elementos consecutivos converge para zero}.
Dado que a sequência de termo geral $q_k$ converge,
então para todo $\varepsilon$ existe um $K$
para o qual $|q_k - L| < \varepsilon$ quando $k \ge K$,
em que $L$ é o limite da sequência.
Essa mesma sequência defasada por um único elemento
irá convergir para o mesmo limite,
bastando fazer um ajuste no índice $k$,
isto é, usando o mesmo $\varepsilon$, sabemos que
$|q_{k+1} - L| < \varepsilon$ quando $k \ge K - 1$.
Quando $k \ge K$, ambas as desigualdades são válidas,
e o limite de $q_{k+1} - q_k$ é zero,
visto que o dobro de um valor arbitrário
continua sendo um valor arbitrário:
\[
  \left.
    \begin{array}{rcl}
        |q_k - L| < \varepsilon
      &\iff&
        - \varepsilon < L - q_k < \varepsilon
      \\
        |q_{k+1} - L| < \varepsilon
      &\iff&
        - \varepsilon < q_{k+1} - L < \varepsilon
    \end{array}
  \right\}
  \;\implies\;
  - 2 \varepsilon < q_{k+1} - q_k < 2 \varepsilon
  \;\iff\;
  |q_{k+1} - q_k| < 2 \varepsilon
\]




\section*{Uma recorrência similar a partir do método de Newton-Raphson}

Seja $f(\alpha) = \alpha^n - x$
em que $\alpha \in \mathds{R}$ e $x, n \in \mathds{N}^*$.
Nota-se que $f(\sqrt[n]{x}) = 0$,
e nosso objetivo é obter esse zero dessa função.
A derivada de $f$ com relação a $\alpha$ é:
\[f'(\alpha) = n \alpha^{n-1}\]
A qual é estritamente positiva para $\alpha$ positivo,
o que significa que $f(\alpha)$ é monotônica crescente
no domínio $\alpha > 0$.
Isso nos diz que a raiz $n$-ésima de $x$
é o único zero real positivo dessa função.

A regra de iteração do método de Newton-Raphson
sobre uma função $f:\mathds{R}\to\mathds{R}$
consiste no uso da raiz (ou zero)
da aproximação linear em torno do ``ponto atual'' da função
como a nova abscissa que será utilizada
para obter o ``ponto seguinte'' da sequência.
Esse processo está ilustrado na figura~\ref{fig:newton-raphson}.

\begin{figure}[H]
  \centering
  \begin{tikzpicture}[
    domain=-.5:15.3,
    scale=.7,
    samples=100,
    axis/.style={very thick, >=stealth', ->},
    grid/.style={thin, color=gray},
    func/.style={thick, color=darkgray},
    approx/.style={thin, color=gray, >=stealth', ->},
  ]

  \draw[grid, dotted, step=1] (0, 0) grid (16, 8);
  \draw[axis] (0, -.5) -- (0, 8.5) node[above] {$f(\alpha)$};
  \draw[axis] (-.5, 0) -- (16.5, 0) node[right] {$\alpha$};
  \draw[func] plot (\x, {(\x / 10) ^ 5 - .2});

  \draw[grid, dashed]
    (15, 7.39375)
      -- (15, 0)
      node[below, color=black] {$\alpha_0$}
    (12.079012345679013, 2.3713259083991125)
      -- (12.079012345679013, 0)
      node[below, color=black] {$\alpha_1$}
    (9.851113126088102, 0.7277405339275622)
      -- (9.851113126088102, 0)
      node[below, color=black] {$\alpha_2$}
    (8.305626200813814, 0.1952409256907805)
      -- (8.305626200813814, 0)
      node[below, color=black] {$\alpha_3$}
    (7.485064339688552, 0.034951211574533014)
      -- (7.485064339688552, 0)
      node[below, color=black] {$\alpha_4$};

  \draw[approx, solid]
    (15, 7.39375)
      -- (12.079012345679013, 0);
  \draw[approx, solid]
    (12.079012345679013, 2.3713259083991125)
      -- (9.851113126088102, 0);
  \draw[approx, solid]
    (9.851113126088102, 0.7277405339275622)
      -- (8.305626200813814, 0);
  \draw[approx, solid]
    (8.305626200813814, 0.1952409256907805)
      -- (7.485064339688552, 0);

  \foreach \x/\xtext in {1, 2, 3, 4, 5, 6, 7, 8, 9, 10,
                         11, 12, 13, 14, 15, 16}
    \draw[shift={(\x, 0)}] (0pt, 2pt) -- (0pt, -2pt) node[below] {};
  \foreach \y in {1, 2, 3, 4, 5, 6, 7, 8}
    \draw[shift={(0, \y)}] (2pt, 0pt) -- (-2pt, 0pt) node[left] {};

\end{tikzpicture}

  \caption{Ilustração do método de Newton-Raphson}
  \label{fig:newton-raphson}
\end{figure}

A aproximação linear\footnote{
  Essa equação é dada na forma $g - g_0 = m (\alpha - \alpha_0)$,
  ou, equivalentemente,
  $g(\alpha) - g(\alpha_0) = m (\alpha - \alpha_0)$,
  em que o coeficiente angular $m$ é a derivada da função $g(\alpha)$.
} em torno da abscissa $\alpha_k$
é a reta tangente à função $f(\alpha)$ nesse ponto,
e é dada por
$g(\alpha_{k+1}) - g(\alpha_k) =
 f'(\alpha_k) \left( \alpha_{k+1} - \alpha_k \right)$,
em que:
\begin{itemize}
  \item
  $g(\alpha_k) = f(\alpha_k)$, pois a aproximação linear
  passa pelo mesmo ponto $(\alpha_k, f(\alpha_k))$ da função; e
  \item
  $g(\alpha_{k+1}) = 0$,
  pois queremos o zero da aproximação linear.
\end{itemize}

Isso permite obter a recorrência
que define a sequência do método de Newton-Raphson:
\[\alpha_{k+1} = \alpha_k - \dfrac{f(\alpha_k)}{f'(\alpha_k)}\]
Aplicando os valores de $f(\alpha)$ e $f'(\alpha)$:
\[
  \begin{array}{rcl}
  \alpha_{k+1}
  &=& \alpha_k - \dfrac{\alpha_k^n - x}{n \alpha_k^{n-1}} \\[5mm]
  &=& \dfrac{\alpha_k n \alpha_k^{n-1}
    - (\alpha_k^n - x)}{n \alpha_k^{n-1}} \\[5mm]
  &=& \dfrac{n \alpha_k^n - \alpha_k^n + x}{n \alpha_k^{n-1}} \\[5mm]
  &=& \dfrac{(n-1) \alpha_k^n + x}{n \alpha_k^{n-1}} \\[5mm]
  &=& \dfrac{(n-1) \alpha_k + \dfrac{x}{\alpha_k^{n-1}}}{n}
  \end{array}
\]
Escrevendo de outra forma:
\[\delta_{k+1} = \dfrac{x}{\alpha_k^{n-1}}\]
\[\alpha_{k+1} = \dfrac{(n-1) \alpha_k + \delta_{k+1}}{n}\]
O que é bastante similar ao algoritmo proposto inicialmente,
a menos da ausência da função chão (ou piso)
que tornava os resultados intermediários naturais/inteiros,
e da ausência, até o momento, de um critério de parada.


\section*{Convergência em $\mathds{R}$ sem truncamentos intermediários}

Esse processo iterativo,
obtido a partir do método de Newton-Raphson,
converge?
Para $n = 1$, basta substituir os valores nas equações
para encontrar $\alpha_1 = x$,
isto é, o algoritmo converge em um único passo,
independente do palpite inicial $\alpha_0$.
E para $n \ge 2$?

Admitindo que $n \ge 2$ e $\alpha_k > 0$,
podemos notar, a partir das equações fornecidas
(soma, multiplicação e divisão de números positivos),
que $\alpha_{k+1} > 0$.
Por indução, sabemos que se o palpite inicial $\alpha_0$ for positivo,
a sequência inteira será formada apenas por números positivos.
Porém, podemos restringir ainda mais
os possíveis valores de $\alpha_{k+1}$.
Para isso, basta lembrar
da desigualdade das médias aritmética e geométrica:
\[
  \dfrac{(n-1) \alpha_k + \delta_{k+1}}{n} \ge
  \sqrt[n]{\alpha_k^{n-1} \delta_{k+1}}
\]
A média aritmética de $n-1$ elementos iguais a $\alpha_k$
e um único elemento $\delta_{k+1}$
é maior que a média geométrica desses mesmos elementos,
a menos que $\alpha_k = \delta_{k+1}$,
situação na qual essas médias são iguais.
Usando as definições de $\delta_{k+1}$ e $\alpha_{k+1}$,
essa mesma desigualdade pode ser escrita como:
\[\alpha_{k+1} \ge \sqrt[n]{x}\]
Isso significa que $\alpha_1 \ge \sqrt[n]{x}$
até mesmo quando $0 < \alpha_0 < \sqrt[n]{x}$,
ou, dito de outra forma,
se a sequência converge para $\alpha_0 \ge \sqrt[n]{x}$,
então ela também converge para $\alpha_0 > 0$.
Se $\alpha_0 = \sqrt[n]{x}$, não haveria o que analisar,
o problema já estaria resolvido
e a sequência se tornaria constante
($\delta_{k+1} = \alpha_{k+1} = \sqrt[n]{x}$,
 $\forall k \in \mathds{N}$).
Mesmo ampliando dessa forma
o conjunto de valores possíveis para o $\alpha_0$,
o único caso que nos interessa é quando $\alpha_0 > \sqrt[n]{x}$.

Se $\alpha_0 > \sqrt[n]{x}$,
então $\alpha_0^n > x$,
o que também pode ser escrito como:
\[\alpha_0 > \dfrac{x}{\alpha_0^{n-1}} = \delta_1\]
Pois só estamos lidando com números positivos.
Porém podemos usar essa desigualdade, $\alpha_0 > \delta_1$,
na definição do $\alpha_1$
para descobrir que $\alpha_1 < \alpha_0$:
\[
  \alpha_1 = \dfrac{(n-1) \alpha_0 + \delta_1}{n}
  < \dfrac{(n-1) \alpha_0 + \alpha_0}{n} = \alpha_0
\]
Isto é, sabemos que $\sqrt[n]{x} \le \alpha_1 < \alpha_0$.
Utilizando exatamente esse mesmo raciocínio,
sabemos que, enquanto não chegarmos ao resultado,
$\sqrt[n]{x} \le \alpha_{k+1} < \alpha_k$,
ou seja, a cada passo/iteração estamos mais próximos do resultado.
Como tal sequência é \emph{limitada} e \emph{monotônica},
sabemos que ela é \emph{convergente}.

Resta identificar
se para qualquer erro máximo tolerável $\varepsilon > 0$
existe um número finito de passos a partir do qual é garantido que
$\alpha_k - \sqrt[n]{x} < \varepsilon$
(i.e., falta mostrar que $\lim_{k\to\infty} \alpha_k = \sqrt[n]{x}$).
Lembrando que:
\[
  \begin{array}{rcl}
  \alpha_{k+1}
  &=& \dfrac{(n-1) \alpha_k + \dfrac{x}{\alpha_k^{n-1}}}{n} \\[5mm]
  &=& \dfrac{(n-1) \alpha_k}{n} + \dfrac{x}{n\alpha_k^{n-1}} \\[5mm]
  &=& \dfrac{(n-1) \alpha_k \alpha_k^{n-1} + x}
            {n\alpha_k^{n-1}} \\[5mm]
  &=& \dfrac{(n-1) \alpha_k^n + x}{n\alpha_k^{n-1}}
  \end{array}
\]
E sabendo que,
para valores grandes de $k$, $\alpha_k \approx \alpha_{k+1}$,
o que pode ser escrito de maneira mais precisa como\footnote{
  Isso é válido pois já foi mostrado que a sequência é convergente.
}:
\[\lim_{k\to\infty} \alpha_{k+1} - \alpha_k = 0\]
Ou, equivalentemente:
\[\lim_{k\to\infty} \alpha_k = \lim_{k\to\infty} \alpha_{k+1} = L\]
Podemos continuar esse equacionamento
sem explicitar os limites para evitar poluir o equacionamento,
de onde obtemos:
\[L = \dfrac{(n-1) L^n + x}{n L^{n-1}}\]
\[L n L^{n-1} = (n-1) L^n + x\]
\[\cancel{nL^n} = \cancel{nL^n} - L^n + x\]
\[L^n = x\]
O que significa\footnote{
  Essa é a única raiz real positiva do polinômio $L^n = x$.
  As $n$ raízes são da forma
  $L = e^{\frac{2\pi i m}{n}} \sqrt[n]{x}$,
  para os diferentes valores inteiros de $m$,
  em que $i$ é a unidade imaginária.
  Para $m = 0$, temos a raiz real positiva.
  Quando $n$ é par, existe uma raiz real negativa,
  obtida com o valor $m = n/2$.
  Todas as demais raízes são complexas
  (possuem uma parcela imaginária).
} que $L = \lim_{k\to\infty} \alpha_k = \sqrt[n]{x}$.

\section*{Corretude do algoritmo em $\mathds{N}^*$
          a partir da aplicação da função chão}

Vamos realizar uma única mudança
no procedimento anteriormente obtido pelo método de Newton-Raphson:
aplicar a função chão após cada passo do algoritmo.
Iniciando com um palpite inteiro $a_0$,
temos uma sequência de palpites
definidos através da recorrência:
\begin{equation}
  a_{k+1}
  = \bracketize[.75em]{\lfloor}{\rfloor}{
      \dfrac{(n-1) a_k + \dfrac{x}{a_k^{n-1}}}{n}
    }
\end{equation}

\subsection*{Função chão interna}

Devido aos resultados \eqref{floor-int} e \eqref{floor-nested-ratio}
relativos à função chão,
podemos reescrever $a_{k+1}$ como:
\begin{equation}
    a_{k+1}
  \stackfirst{\eqref{floor-nested-ratio}}{=}
    \bracketize[.75em]{\lfloor}{\rfloor}{
      \dfrac{\left\lfloor
               (n-1) a_k + \dfrac{x}{a_k^{n-1}}
             \right\rfloor}
            {n}
    }
  \stackfirst{\eqref{floor-int}}{=}
    \bracketize[.75em]{\lfloor}{\rfloor}{
      \dfrac{(n-1) a_k +
             \left\lfloor \dfrac{x}{a_k^{n-1}} \right\rfloor}
            {n}
    }
\end{equation}
Pois o denominador $n$ é um inteiro positivo,
e a parcela $(n - 1) a_k$ é inteira.
Essa nova formulação corresponde precisamente
ao algoritmo inicialmente proposto:
\[
  a_{k+1} = \left\lfloor \dfrac{(n-1) a_k + d_{k+1}}{n} \right\rfloor,
  \quad\text{onde}\quad
  d_{k+1} = \left\lfloor \dfrac{x}{a_k^{n-1}} \right\rfloor \\
\]

Isso significa que a aplicação da função chão interna
na definição da recorrência $a_k$ não altera em nada o valor de $a_k$,
desde que a função chão externa seja mantida.

\subsection*{Limitante inferior e valor inicial da sequência}

A desigualdade das médias aritmética e geométrica
pode ser aplicada à média de $n - 1$ ocorrências de $a_k$
e uma ocorrência de $x/a_k^{n-1}$:
\[
  \dfrac{(n-1) a_k + \dfrac{x}{a_k^{n-1}}}{n} \ge
  \sqrt[n]{a_k^{n-1} \dfrac{x}{a_k^{n-1}}}
\]
Aplicando a função chão em cada lado da inequação\footnote{
  Dado $\gamma \le \beta$, sabemos que
  $\lfloor \gamma \rfloor \le \beta$,
  pois $\lfloor \gamma \rfloor \le \gamma$
  pela definição da função chão.
  Aplicando o resultado \eqref{floor-switch-ge}
  à inequação $\lfloor \gamma \rfloor \le \beta$
  fornece $\lfloor \gamma \rfloor \le \lfloor \beta \rfloor$.
  Analogamente, dado $\gamma < \beta$,
  sabemos que $\lfloor \gamma \rfloor < \beta$
  pela definição da função chão,
  e aplicando o resultado \eqref{floor-lt2ge-int}
  obtemos $\lfloor \gamma \rfloor \le \lfloor \beta \rfloor$.
  Resumidamente,
  é sempre possível aplicar a função chão
  nos dois lados de uma inequação,
  mas a inequação do resultado deverá ser não-estrita.
  O mesmo resultado pode ser interpretado
  como consequência do fato
  de que a função chão é monotônica não-decrescente,
  mas não é estritamente crescente.
},
temos:
\[
  a_{k+1} \ge \lfloor \sqrt[n]{x} \rfloor
\]
De forma que, basta termos $a_0 > 0$
para garantirmos $a_k \ge \lfloor \sqrt[n]{x} \rfloor$ para $k > 0$.
Mas seria interessante adotar uma condição inicial
que também mantivesse essa propriedade.
Para isso, suponha que $a_s = 1$, com isso teríamos:
\[
  a_{s+1}
  = \left\lfloor \dfrac{n - 1 + x}{n} \right\rfloor
  = \left\lfloor \dfrac{x - 1}{n} + 1 \right\rfloor
  \stackfirst{\eqref{floor-int}}{=}
    \left\lfloor \dfrac{x - 1}{n} \right\rfloor + 1
\]
Isso significa que:
\[
  \left\lfloor \dfrac{x - 1}{n} \right\rfloor + 1
  \ge \lfloor \sqrt[n]{x} \rfloor
\]
Ou seja, caso adotemos como condição inicial:
\[
  a_0 = \left\lfloor \dfrac{x - 1}{n} \right\rfloor + 1
\]
Garantimos $a_k \ge \lfloor \sqrt[n]{x} \rfloor$
para todo $k \in \mathds{N}$.

\subsection*{Tamanho do passo}

Podemos reorganizar o valor de $a_{k+1}$ da seguinte forma:
\[
  a_{k+1}
  = \bracketize[.75em]{\lfloor}{\rfloor}{
      \dfrac{(n-1) a_k + \dfrac{x}{a_k^{n-1}}}{n}
    }
  = \bracketize[.75em]{\lfloor}{\rfloor}{
      \dfrac{n a_k - a_k + \dfrac{x}{a_k^{n-1}}}{n}
    }
  = \bracketize[.75em]{\lfloor}{\rfloor}{
      a_k +
      \dfrac{- \dfrac{a_k^n}{a_k^{n-1}} + \dfrac{x}{a_k^{n-1}}}
            {n}
    }
  = \left\lfloor
      a_k +
      \dfrac{x - a_k^n}{na_k^{n-1}}
    \right\rfloor
\]
Como $a_k \in \mathds{N}$, podemos retirar da função chão:
\[
  a_{k+1}
  = a_k +
    \left\lfloor
      \dfrac{x - a_k^n}{na_k^{n-1}}
    \right\rfloor
\]
De forma que o passo de atualização $p_{k+1} = a_{k+1} - a_{k}$
é dado por:
\[
  p_{k+1} = \left\lfloor \dfrac{x - a_k^n}{na_k^{n-1}} \right\rfloor
\]

\subsection*{Sinal do passo e condição de parada}

Utilizando os resultados acerca da função chão
apresentados em \eqref{floor-switch-lt} e \eqref{floor-switch-ge},
podemos relacionar o sinal dos valores de $a_k$, $p_{k+1}$ e $d_{k+1}$:
\[
  \overbrace{
    \left\lfloor \dfrac{x - a_k^n}{na_k^{n-1}} \right\rfloor
  }^{p_{k+1}} < 0
  \;\stackfirst{\eqref{floor-switch-lt}}{\iff}\;
  \dfrac{x - a_k^n}{na_k^{n-1}} < 0
  \;\stackfirst{a_k > 0}{
      \underset{n > 0\vphantom{\int}}{\iff}
    }\;
  x - a_k^n < 0
  \;\iff\;
  \overset{
    \mathclap{\substack{
      \displaystyle a_k > \sqrt[n]{x} \\[.5em]
      \displaystyle \Big\Updownarrow \\[.5em]
    }}
  }{a_k^n > x}
  \;\stackfirst{a_k > 0}{\iff}\;
  a_k > \dfrac{x}{a_k^{n-1}}
  \;\underset{\eqref{floor-switch-lt}}{
      \stackfirst{a_k \in \mathds{N}}{\iff}
    }\;
  a_k > \overbrace{
          \left\lfloor \dfrac{x}{a_k^{n-1}} \right\rfloor
        }^{d_{k+1}}
\]
\[
  \underbrace{
    \left\lfloor \dfrac{x - a_k^n}{na_k^{n-1}} \right\rfloor
  }_{p_{k+1}} \ge 0
  \;\stackfirst{\eqref{floor-switch-ge}}{\iff}\;
  \dfrac{x - a_k^n}{na_k^{n-1}} \ge 0
  \;\stackfirst{a_k > 0}{
      \underset{n > 0\vphantom{\int}}{\iff}
    }\;
  x - a_k^n \ge 0
  \;\iff\;
  \underset{
    \mathclap{\substack{\\[.4em]
      \displaystyle \Big\Updownarrow \\[.5em]
      \displaystyle a_k \le \sqrt[n]{x}
    }}
  }{a_k^n \le x}
  \;\stackfirst{a_k > 0}{\iff}\;
  a_k \le \dfrac{x}{a_k^{n-1}}
  \;\underset{\eqref{floor-switch-ge}}{
      \stackfirst{a_k \in \mathds{N}}{\iff}
    }\;
  a_k \le \underbrace{
            \left\lfloor \dfrac{x}{a_k^{n-1}} \right\rfloor
          }_{d_{k+1}}
\]
Tal resultado pode ser resumido como:
\begin{equation}
  \begin{array}{lcccr}
    p_{k+1}   < 0 &\iff& a_k   > \sqrt[n]{x} &\iff& a_k   > d_{k+1} \\
    p_{k+1} \ge 0 &\iff& a_k \le \sqrt[n]{x} &\iff& a_k \le d_{k+1}
  \end{array}
\end{equation}
Resta ainda entender melhor o cenário em que $a_k \le \sqrt[n]{x}$,
bem como a forma com a qual esse resultado
se relaciona com a corretude do algoritmo e o critério de parada.

Suponha $\lfloor \sqrt[n]{x} \rfloor \le a_k < \sqrt[n]{x}$.
Pela definição da função chão, temos
$\sqrt[n]{x} < \lfloor\sqrt[n]{x}\rfloor + 1$.
Juntar as desigualdades fornece
$\lfloor\sqrt[n]{x}\rfloor \le a_k < \lfloor\sqrt[n]{x}\rfloor + 1$,
o que nos obriga a dizer que $a_k = \lfloor\sqrt[n]{x}\rfloor$
pelo fato de $a_k$ ser sempre um número inteiro
$a_k = \lfloor a_k \rfloor$,
e pelas últimas desigualdades consistirem
na própria definição da função chão.
Isto é:
\begin{equation}\label{ak-between-floor-and-root}
  \lfloor \sqrt[n]{x} \rfloor \le a_k < \sqrt[n]{x}
  \quad\implies\quad
  a_k = \lfloor \sqrt[n]{x} \rfloor
\end{equation}
Por outro lado, suponha $a_k = \sqrt[n]{x}$.
Como $a_k \in \mathds{N}^*$,
também sabemos que $\sqrt[n]{x} = \lfloor \sqrt[n]{x} \rfloor$.
Isso significa que
quando $\lfloor \sqrt[n]{x} \rfloor \le a_k \le \sqrt[n]{x}$,
então $a_k = \lfloor \sqrt[n]{x} \rfloor$.
A desigualdade entre as médias aritmética e geométrica
nos mostrou que $a_{k+1} \ge \lfloor \sqrt[n]{x} \rfloor$,
como explícito na inequação \eqref{amgm-min-ak+1}.
Isso nos permite reescrever
a relação entre o sinal dos valores de $a_k$, $p_{k+1}$ e $d_{k+1}$
como uma variante mais restrita,
admindo que $k > 0$ ou $a_0 \ge \lfloor \sqrt[n]{x} \rfloor$:
\begin{equation}
  \begin{array}{lcccr}
      p_{k+1} < 0
    &\iff&
      a_k > \sqrt[n]{x}
    &\iff&
      a_k > d_{k+1}
    \\
      p_{k+1} \ge 0
    &\underset{\substack{k > 0 \text{ ou} \\[.2em]
                         a_0 \ge \lfloor \sqrt[n]{x} \rfloor}}{\iff}&
      a_k = \lfloor \sqrt[n]{x} \rfloor
    &\underset{\substack{k > 0 \text{ ou} \\[.2em]
                         a_0 \ge \lfloor \sqrt[n]{x} \rfloor}}{\iff}&
      a_k \le d_{k+1}
  \end{array}
\end{equation}
Dessa forma,
podemos considerar que o critério de parada é $p_{k+1} \ge 0$.
Em outras palavras,
\emph{o processo iterativo deve continuar
      apenas enquanto $a_k$ estiver diminuindo},
e quando parar de diminuir,
\emph{o valor mínimo encontrado é o próprio resultado desejado}.
Como estamos lidando apenas com valores inteiros,
isso também significa que o resultado é obtido
após um número finito de iterações.

Equivalentemente, para não ser necessário calcular o passo,
a condição de parada é $a_k \le d_{k+1}$,
o que corresponde à descrição do algoritmo,
finalizando a prova de sua corretude.



\end{document}
