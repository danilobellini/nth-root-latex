\documentclass{article}

\usepackage{mainstyle}

\title{Algorithm to numerically compute $\lfloor \sqrt[n]{x} \rfloor$
       given $x, n \in \mathds{N}^*$}

\author{Danilo J. S. Bellini}


\begin{document}

\maketitle

\section*{Abstract}

In PEP 572, there's an algorithm\footnote{
  \url{https://www.python.org/dev/peps/pep-0572/\#a-numeric-example}
}
implemented in Python with and without an \emph{assignment expression}
for obtaining the $n$-th root of $x$,
using only integer numbers (input, output and in-between values),
where the final result is the truncated value of the desired root
if it's not an integer.
The purpose of this text is to explain, justify, prove and analyze
this algorithm and how it works.



\section*{Descrição do algoritmo}

Sejam $x$ e $n$ valores inteiros positivos,
e $a_0 \ge \sqrt[n]{x}$ um palpite inicial, também inteiro e positivo
(por exemplo, o próprio valor $x$).
Para encontrar numericamente o valor de $\lfloor \sqrt[n]{x} \rfloor$,
calcularemos iterativamente a seguinte sequência
enquanto $a_k > d_{k+1}$:
\[d_{k+1} = \left\lfloor \dfrac{x}{a_k^{n-1}} \right\rfloor\]
\[a_{k+1} = \left\lfloor \dfrac{(n-1) a_k + d_{k+1}}{n} \right\rfloor\]
O resultado é o último valor $a_k$ encontrado nesse processo.
Isto é, o resultado é o valor de $a_r$,
em que $r$ é menor índice $k$ para o qual vale $a_k \le d_{k+1}$.

Esse algoritmo pode ser adaptado
a partir da forma\footnote{
  O uso dessa sintaxe está disponível a partir do Python 3.8.
} como consta na PEP 572
para explicitar um valor inicial padrão e ser escrito como uma função:

\begin{center}
  \begin{minipage}{7cm}
    \inputminted{python}{nth_root.py}
  \end{minipage}
\end{center}


\section*{Uma recorrência similar a partir do método de Newton-Raphson}

Seja $f(\alpha) = \alpha^n - x$
em que $\alpha \in \mathds{R}$ e $x, n \in \mathds{N}^*$.
Nota-se que $f(\sqrt[n]{x}) = 0$,
e nosso objetivo é obter esse zero dessa função.
A derivada de $f$ com relação a $\alpha$ é:
\[f'(\alpha) = n \alpha^{n-1}\]
A qual é estritamente positiva para $\alpha$ positivo,
o que significa que $f(\alpha)$ é monotônica crescente
no domínio $\alpha > 0$.
Isso nos diz que a raiz $n$-ésima de $x$
é o único zero real positivo dessa função.

A regra de iteração do método de Newton-Raphson
sobre uma função $f:\mathds{R}\to\mathds{R}$
consiste no uso da raiz (ou zero)
da aproximação linear em torno do ``ponto atual'' da função
como a nova abscissa que será utilizada
para obter o ``ponto seguinte'' da sequência.
Esse processo está ilustrado na figura~\ref{fig:newton-raphson}.

\begin{figure}[H]
  \centering
  \begin{tikzpicture}[
    domain=-.5:15.3,
    scale=.7,
    samples=100,
    axis/.style={very thick, >=stealth', ->},
    grid/.style={thin, color=gray},
    func/.style={thick, color=darkgray},
    approx/.style={thin, color=gray, >=stealth', ->},
  ]

  \draw[grid, dotted, step=1] (0, 0) grid (16, 8);
  \draw[axis] (0, -.5) -- (0, 8.5) node[above] {$f(\alpha)$};
  \draw[axis] (-.5, 0) -- (16.5, 0) node[right] {$\alpha$};
  \draw[func] plot (\x, {(\x / 10) ^ 5 - .2});

  \draw[grid, dashed]
    (0, 7.39375) node[left, color=black] {$f(\alpha_0)$} --
    (15, 7.39375)
      -- (15, 0)
      node[below, color=black] {$\alpha_0$}
    (0, 2.3713259083991125) node[left, color=black] {$f(\alpha_1)$} --
    (12.079012345679013, 2.3713259083991125)
      -- (12.079012345679013, 0)
      node[below, color=black] {$\alpha_1$}
    (0, 0.7277405339275622) node[left, color=black] {$f(\alpha_2)$} --
    (9.851113126088102, 0.7277405339275622)
      -- (9.851113126088102, 0)
      node[below, color=black] {$\alpha_2$}
    (0, 0.1952409256907805) --
    (8.305626200813814, 0.1952409256907805)
      -- (8.305626200813814, 0)
      node[below, color=black] {$\alpha_3$}
    (0, 0.034951211574533014) --
    (7.485064339688552, 0.034951211574533014)
      -- (7.485064339688552, 0)
      node[below, color=black] {$\alpha_4$};

  \draw[approx, solid]
    (15, 7.39375)
      -- (12.079012345679013, 0);
  \draw[approx, solid]
    (12.079012345679013, 2.3713259083991125)
      -- (9.851113126088102, 0);
  \draw[approx, solid]
    (9.851113126088102, 0.7277405339275622)
      -- (8.305626200813814, 0);
  \draw[approx, solid]
    (8.305626200813814, 0.1952409256907805)
      -- (7.485064339688552, 0);

  \foreach \x in {1, ..., 16}
    \draw[shift={(\x, 0)}] (0pt, 2pt) -- (0pt, -2pt) node[below] {};
  \foreach \y in {1, ..., 8}
    \draw[shift={(0, \y)}] (2pt, 0pt) -- (-2pt, 0pt) node[left] {};

\end{tikzpicture}

  \caption{Ilustração do método de Newton-Raphson}
  \label{fig:newton-raphson}
\end{figure}

A aproximação linear\footnote{
  Essa equação é dada na forma $g - g_0 = m (\alpha - \alpha_0)$,
  ou, equivalentemente,
  $g(\alpha) - g(\alpha_0) = m (\alpha - \alpha_0)$,
  em que o coeficiente angular $m$ é a derivada da função $g(\alpha)$.
} em torno da abscissa $\alpha_k$
é a reta tangente à função $f(\alpha)$ nesse ponto,
e é dada por
$g(\alpha_{k+1}) - g(\alpha_k) =
 f'(\alpha_k) \left( \alpha_{k+1} - \alpha_k \right)$,
em que:
\begin{itemize}
  \item
  $g(\alpha_k) = f(\alpha_k)$, pois a aproximação linear
  passa pelo mesmo ponto $(\alpha_k, f(\alpha_k))$ da função; e
  \item
  $g(\alpha_{k+1}) = 0$,
  pois queremos o zero da aproximação linear.
\end{itemize}

Isso permite obter a recorrência
que define a sequência do método de Newton-Raphson:
\[\alpha_{k+1} = \alpha_k - \dfrac{f(\alpha_k)}{f'(\alpha_k)}\]
Aplicando os valores de $f(\alpha)$ e $f'(\alpha)$:
\[
  \begin{array}{rcl}
  \alpha_{k+1}
  &=& \alpha_k - \dfrac{\alpha_k^n - x}{n \alpha_k^{n-1}} \\[5mm]
  &=& \dfrac{\alpha_k n \alpha_k^{n-1}
    - (\alpha_k^n - x)}{n \alpha_k^{n-1}} \\[5mm]
  &=& \dfrac{n \alpha_k^n - \alpha_k^n + x}{n \alpha_k^{n-1}} \\[5mm]
  &=& \dfrac{(n-1) \alpha_k^n + x}{n \alpha_k^{n-1}} \\[5mm]
  &=& \dfrac{(n-1) \alpha_k + \dfrac{x}{\alpha_k^{n-1}}}{n}
  \end{array}
\]
Escrevendo de outra forma:
\[\delta_{k+1} = \dfrac{x}{\alpha_k^{n-1}}\]
\[\alpha_{k+1} = \dfrac{(n-1) \alpha_k + \delta_{k+1}}{n}\]
O que é bastante similar ao algoritmo proposto inicialmente,
a menos da ausência da função chão (ou piso)
que tornava os resultados intermediários naturais/inteiros,
e da ausência, até o momento, de um critério de parada.


\section*{Convergência em $\mathds{R}$ sem truncamentos intermediários}

Esse processo iterativo,
obtido a partir do método de Newton-Raphson,
converge?
Para $n = 1$, basta substituir os valores nas equações
para encontrar $\alpha_1 = x$,
isto é, o algoritmo converge em um único passo,
independente do palpite inicial $\alpha_0$.
E para $n \ge 2$?

Admitindo que $n \ge 2$ e $\alpha_k > 0$,
podemos notar, a partir das equações fornecidas
(soma, multiplicação e divisão de números positivos),
que $\alpha_{k+1} > 0$.
Por indução, sabemos que se o palpite inicial $\alpha_0$ for positivo,
a sequência inteira será formada apenas por números positivos.
Porém, podemos restringir ainda mais
os possíveis valores de $\alpha_{k+1}$.
Para isso, basta lembrar
da desigualdade das médias aritmética e geométrica\footnote{
  É possível provar a desigualdade
  $\sum_{i=1}^n y_i \ge n \sqrt[n]{\prod_{i=1}^n y_i}$,
  por meio de duas induções em direções opostas.
  O caso base é:
  \[
    (y_1 - y_2)^2 \ge 0
    \implies y_1^2 - 2 y_1 y_2 + y_2^2 \ge 0
    \implies y_1^2 + 2 y_1 y_2 + y_2^2 \ge 4 y_1 y_2
    \implies (y_1 + y_2)^2 \ge 4 y_1 y_2
    \implies y_1 + y_2 \ge 2 \sqrt{y_1 y_2}
  \]
  Se a desigualdade vale para $n$ elementos,
  ela também valerá para $2n$ elementos
  (prova para as potências de $2$):
  \[
    \sum_{i=1}^{2n} y_i
    = \sum_{i=1}^{n} y_i + \sum_{i=n+1}^{2n} y_i
    \ge n \left( \sqrt[n]{\prod_{i=1}^n y_i}
               + \sqrt[n]{\prod_{i=n+1}^{2n} y_i} \right)
    \ge n 2 \sqrt{ \left( \sqrt[n]{\prod_{i=1}^n y_i} \right)
                   \left( \sqrt[n]{\prod_{i=n+1}^{2n} y_i} \right) }
    = 2n \sqrt[2n]{\prod_{i=1}^{2n} y_i}
  \]
  Se tal desigualdade vale para $n$ elementos,
  ela também valerá para $n-1$ elementos.
  Considere $y_n = \dfrac{1}{n-1} \sum_{i=1}^{n-1} y_i$:
  \[
    \sum_{i=1}^{n} y_i
    = \sum_{i=1}^{n-1} y_i + \dfrac{1}{n-1} \sum_{i=1}^{n-1} y_i
    = \left( 1 + \dfrac{1}{n-1} \right) \sum_{i=1}^{n-1} y_i
    = \dfrac{n}{n-1} \sum_{i=1}^{n-1} y_i
    \ge n \sqrt[n]{\prod_{i=1}^n y_i}
    = n \sqrt[n]{\prod_{i=1}^{n-1} y_i
                 \left( \dfrac{1}{n-1} \sum_{j=1}^{n-1} y_j \right)}
  \]
  Dividindo por $n$ cada lado da desigualdade e elevando a $n$:
  \[
    \left( \dfrac{1}{n-1}
           \sum_{i=1}^{n-1} y_i \right)^{\cancelto{n-1}{n}}
    \ge \prod_{i=1}^{n-1} y_i
        \cancel{\left( \dfrac{1}{n-1} \sum_{j=1}^{n-1} y_j \right)}
    \implies
    \sum_{i=1}^{n-1} y_i \ge (n-1) \sqrt[n-1]{\prod_{i=1}^{n-1} y_i}
  \]
  \hfill$\blacksquare$
}:
\[
  \dfrac{(n-1) \alpha_k + \delta_{k+1}}{n} \ge
  \sqrt[n]{\alpha_k^{n-1} \delta_{k+1}}
\]
A média aritmética de $n-1$ elementos iguais a $\alpha_k$
e um único elemento $\delta_{k+1}$
é maior que a média geométrica desses mesmos elementos,
a menos que $\alpha_k = \delta_{k+1}$,
situação na qual essas médias são iguais.
Usando as definições de $\delta_{k+1}$ e $\alpha_{k+1}$,
essa mesma desigualdade pode ser escrita como:
\[\alpha_{k+1} \ge \sqrt[n]{x}\]
Isso significa que $\alpha_1 \ge \sqrt[n]{x}$
até mesmo quando $0 < \alpha_0 < \sqrt[n]{x}$,
ou, dito de outra forma,
se a sequência converge para $\alpha_0 \ge \sqrt[n]{x}$,
então ela também converge para $\alpha_0 > 0$.
Se $\alpha_0 = \sqrt[n]{x}$, não haveria o que analisar,
o problema já estaria resolvido
e a sequência se tornaria constante
($\delta_{k+1} = \alpha_{k+1} = \sqrt[n]{x}$,
 $\forall k \in \mathds{N}$).
Mesmo ampliando dessa forma
o conjunto de valores possíveis para o $\alpha_0$,
o único caso que nos interessa é quando $\alpha_0 > \sqrt[n]{x}$.

Se $\alpha_0 > \sqrt[n]{x}$,
então $\alpha_0^n > x$,
o que também pode ser escrito como:
\[\alpha_0 > \dfrac{x}{\alpha_0^{n-1}} = \delta_1\]
Pois só estamos lidando com números positivos.
Porém podemos usar essa desigualdade, $\alpha_0 > \delta_1$,
na definição do $\alpha_1$
para descobrir que $\alpha_1 < \alpha_0$:
\[
  \alpha_1 = \dfrac{(n-1) \alpha_0 + \delta_1}{n}
  < \dfrac{(n-1) \alpha_0 + \alpha_0}{n} = \alpha_0
\]
Isto é, sabemos que $\sqrt[n]{x} \le \alpha_1 < \alpha_0$.
Utilizando exatamente esse mesmo raciocínio,
sabemos que, enquanto não chegarmos ao resultado,
$\sqrt[n]{x} \le \alpha_{k+1} < \alpha_k$,
ou seja, a cada passo/iteração estamos mais próximos do resultado.
Como tal sequência é \emph{limitada} e \emph{monotônica},
sabemos que ela é \emph{convergente}.

Resta identificar
se para qualquer erro máximo tolerável $\varepsilon > 0$
existe um número finito de passos a partir do qual é garantido que
$\alpha_k - \sqrt[n]{x} < \varepsilon$
(i.e., falta mostrar que $\lim_{k\to\infty} \alpha_k = \sqrt[n]{x}$).
Lembrando que:
\[
  \begin{array}{rcl}
  \alpha_{k+1}
  &=& \dfrac{(n-1) \alpha_k + \dfrac{x}{\alpha_k^{n-1}}}{n} \\[5mm]
  &=& \dfrac{(n-1) \alpha_k}{n} + \dfrac{x}{n\alpha_k^{n-1}} \\[5mm]
  &=& \dfrac{(n-1) \alpha_k \alpha_k^{n-1} + x}
            {n\alpha_k^{n-1}} \\[5mm]
  &=& \dfrac{(n-1) \alpha_k^n + x}{n\alpha_k^{n-1}}
  \end{array}
\]
E sabendo que,
para valores grandes de $k$, $\alpha_k \approx \alpha_{k+1}$,
o que pode ser escrito de maneira mais precisa como\footnote{
  Isso é válido pois já foi mostrado que a sequência é convergente.
}:
\[\lim_{k\to\infty} \alpha_{k+1} - \alpha_k = 0\]
Ou, equivalentemente:
\[\lim_{k\to\infty} \alpha_k = \lim_{k\to\infty} \alpha_{k+1} = L\]
Podemos continuar esse equacionamento
sem explicitar os limites para evitar poluir o equacionamento,
de onde obtemos:
\[L = \dfrac{(n-1) L^n + x}{n L^{n-1}}\]
\[L n L^{n-1} = (n-1) L^n + x\]
\[\cancel{nL^n} = \cancel{nL^n} - L^n + x\]
\[L^n = x\]
O que significa\footnote{
  Essa é a única raiz real positiva do polinômio $L^n = x$.
  As $n$ raízes são da forma
  $L = e^{\frac{2\pi i m}{n}} \sqrt[n]{x}$,
  para os diferentes valores inteiros de $m$,
  em que $i$ é a unidade imaginária.
  Para $m = 0$, temos a raiz real positiva.
  Quando $n$ é par, existe uma raiz real negativa,
  obtida com o valor $m = n/2$.
  Todas as demais raízes são complexas
  (possuem uma parcela imaginária).
} que $L = \lim_{k\to\infty} \alpha_k = \sqrt[n]{x}$.

\section*{Convergência em $\mathds{N}^*$
          a partir da aplicação da função chão}

Vamos realizar uma única mudança
no procedimento anteriormente obtido pelo método de Newton-Raphson:
aplicar a função chão após cada passo do algoritmo.
Iniciando com um palpite inteiro $a_0$,
temos uma sequência de palpites
definidos através da recorrência:
\[
  a_{k+1}
  = \bracketize[.75em]{\lfloor}{\rfloor}{
      \dfrac{(n-1) a_k + \dfrac{x}{a_k^{n-1}}}{n}
    }
\]

\subsection*{Função chão interna}

Devido às propriedades da função chão,
podemos reescrever $a_{k+1}$ como:
\[
  a_{k+1}
  = \bracketize[.75em]{\lfloor}{\rfloor}{
      \dfrac{\left\lfloor
               (n-1) a_k + \dfrac{x}{a_k^{n-1}}
             \right\rfloor}
            {n}
    } \\
  = \bracketize[.75em]{\lfloor}{\rfloor}{
      \dfrac{(n-1) a_k +
             \left\lfloor \dfrac{x}{a_k^{n-1}} \right\rfloor}
            {n}
    } \\
\]
Pois o denominador $n$ é um inteiro positivo,
e a parcela $(n - 1) a_k$ é inteira.
Essa nova formulação corresponde precisamente
ao algoritmo inicialmente proposto:
\[
  a_{k+1} = \left\lfloor \dfrac{(n-1) a_k + d_{k+1}}{n} \right\rfloor,
  \quad\text{onde}\quad
  d_{k+1} = \left\lfloor \dfrac{x}{a_k^{n-1}} \right\rfloor \\
\]

Isso significa que a aplicação da função chão interna
na definição da recorrência $a_k$ não altera em nada o falor de $a_k$,
desde que a função chão externa seja mantida.

\subsection*{Limitante inferior e valor inicial da sequência}

A desigualdade das médias aritmética e geométrica
pode ser aplicada à média de $n - 1$ ocorrências de $a_k$
e uma ocorrência de $x/a_k^{n-1}$:
\[
  \dfrac{(n-1) a_k + \dfrac{x}{a_k^{n-1}}}{n} \ge
  \sqrt[n]{a_k^{n-1} \dfrac{x}{a_k^{n-1}}}
\]
Aplicando a função chão de cada lado da inequação, temos, de imediato:
\[
  a_{k+1} \ge \lfloor \sqrt[n]{x} \rfloor
\]
De forma que, basta termos $a_0 > 0$
para garantirmos $a_k \ge \lfloor \sqrt[n]{x} \rfloor$ para $k > 0$.
Mas seria interessante adotar uma condição inicial
que também mantivesse essa propriedade.
Para isso, suponha que $a_s = 1$, com isso teríamos:
\[
  a_{s+1}
  = \left\lfloor \dfrac{n - 1 + x}{n} \right\rfloor
  = \left\lfloor \dfrac{x - 1}{n} + 1 \right\rfloor
  = \left\lfloor \dfrac{x - 1}{n} \right\rfloor + 1
\]
Isso significa que:
\[
  \left\lfloor \dfrac{x - 1}{n} \right\rfloor + 1
  \ge \lfloor \sqrt[n]{x} \rfloor
\]
Ou seja, caso adotemos como condição inicial:
\[
  a_0 = \left\lfloor \dfrac{x - 1}{n} \right\rfloor + 1
\]
Garantimos $a_k \ge \lfloor \sqrt[n]{x} \rfloor$
para todo $k \in \mathds{N}$.

\subsection*{Tamanho do passo}

Podemos reorganizar o valor de $a_{k+1}$ da seguinte forma:
\[
  a_{k+1}
  = \bracketize[.75em]{\lfloor}{\rfloor}{
      \dfrac{(n-1) a_k + \dfrac{x}{a_k^{n-1}}}{n}
    }
  = \bracketize[.75em]{\lfloor}{\rfloor}{
      \dfrac{n a_k - a_k + \dfrac{x}{a_k^{n-1}}}{n}
    }
  = \bracketize[.75em]{\lfloor}{\rfloor}{
      a_k +
      \dfrac{- \dfrac{a_k^n}{a_k^{n-1}} + \dfrac{x}{a_k^{n-1}}}
            {n}
    }
  = \left\lfloor
      a_k +
      \dfrac{x - a_k^n}{na_k^{n-1}}
    \right\rfloor
\]
Como $a_k \in \mathds{N}$, podemos retirar da função chão:
\[
  a_{k+1}
  = a_k +
    \left\lfloor
      \dfrac{x - a_k^n}{na_k^{n-1}}
    \right\rfloor
\]
De forma que o passo de atualização $p_{k+1} = a_{k+1} - a_{k}$
é dado por:
\[
  p_{k+1} = \left\lfloor \dfrac{x - a_k^n}{na_k^{n-1}} \right\rfloor
\]



\end{document}
