\subsection*{Ceiling function and its relation to the floor function}

The ceiling function can be defined
as the $\lceil \cdot \rceil : \mathds{R} \to \mathds{Z}$ function
that satisfies:
\begin{equation}\tag{Ceiling}
  \lceil \gamma \rceil - 1 < \gamma \le \lceil \gamma \rceil
\end{equation}
That is,
$\lceil \gamma \rceil$ is the least integer
greater than or equal to $\gamma$.

Based on the definitions of the floor and ceiling functions,
when $\gamma$ is integer, we have:
\begin{equation}\label{ceil-floor-z}
  \lfloor \gamma \rfloor = \gamma = \lceil \gamma \rceil
  \quad \forall \gamma \in \mathds{Z}
\end{equation}
And, in the complementary scenario, when $\gamma \notin \mathds{Z}$,
we can state that the inequalities of the definitions aren't strict;
more precisely,
$\lfloor \gamma \rfloor < \gamma < \lfloor \gamma \rfloor + 1$
and $\lceil \gamma \rceil - 1 < \gamma < \lceil \gamma \rceil$.
Applying the transitive property, we can obtain two new inequations:
$\lfloor \gamma \rfloor < \lceil \gamma \rceil$
and $\lceil \gamma \rceil - 1 < \lfloor \gamma \rfloor + 1$.
Result \eqref{int+1}
applied to each of these two inequations yields
$\lfloor \gamma \rfloor + 1 \le \lceil \gamma \rceil$ and
$\lceil \gamma \rceil \le \lfloor \gamma \rfloor + 1$, respectively.
This justifies the equation
$\lceil \gamma \rceil = \lfloor \gamma \rfloor + 1$,
which can be rewritten as:
\begin{equation}\label{ceil-floor-notz}
  \lceil \gamma \rceil - \lfloor \gamma \rfloor = 1
  \quad \forall \gamma \in \mathds{R} \setminus \mathds{Z}
\end{equation}
We can apply
these last \eqref{ceil-floor-z} and \eqref{ceil-floor-notz} results
amid proofs of expressions
that simultaneously involve both floor and ceiling functions,
partitioning between integers and non-integers
for proving each case separately.
This recourse will be applied in what follows.

Let $x \in \mathds{Z}$ and $n \in \mathds{N}^*$.
When $x/n \in \mathds{Z}$, we can write:
\[
    \left\lfloor \dfrac{x - 1}{n} \right\rfloor =
    \left\lfloor \dfrac{x}{n} - \dfrac{1}{n} \right\rfloor
      \stackfirst{\text{\eqref{floor-int}}}{=}
    \dfrac{x}{n} + \left\lfloor - \dfrac{1}{n} \right\rfloor =
    \dfrac{x}{n} - 1
  \quad\implies\quad
    \left\lfloor \dfrac{x - 1}{n} \right\rfloor + 1 =
    \dfrac{x}{n}
      \stackfirst{(\mathds{Z})}{=}
    \left\lceil \dfrac{x\vphantom{l}}{n} \right\rceil
\]
By applying the property described in equation \eqref{floor-int}
and from the fact that $\lfloor -1/n \rfloor = -1$,
since $-1 \le -1/n < 0$.
On the other hand, when $x/n \notin \mathds{Z}$,
starting from the definition of the floor function,
already discarding the equality case as it's only for integers,
and based on the results \eqref{int+1} and \eqref{floor-switch-ge},
we can write:
\begin{align*}
    \left\lfloor \dfrac{x\vphantom{l}}{n} \right\rfloor
    < \dfrac{x\vphantom{l}}{n}
  &\quad\stackrel{\rule[-.3em]{0pt}{0pt}n > 0}{\iff}\quad
    n \left\lfloor \dfrac{x\vphantom{l}}{n} \right\rfloor < x
  \\&\quad\iff\quad
    0 < x - n \left\lfloor \dfrac{x\vphantom{l}}{n} \right\rfloor
  \\&\quad\stackrel{\rule[-.3em]{0pt}{0pt}\eqref{int+1}}{\iff}\quad
    1 \le x - n \left\lfloor \dfrac{x\vphantom{l}}{n} \right\rfloor
  \\&\quad\iff\quad
    n \left\lfloor \dfrac{x\vphantom{l}}{n} \right\rfloor \le x - 1
  \\&\quad\stackrel{\rule[-.3em]{0pt}{0pt}n > 0}{\iff}\quad
    \left\lfloor \dfrac{x\vphantom{l}}{n} \right\rfloor
    \le \dfrac{x - 1}{n}
  \\&\quad\stackrel{\rule[-.3em]{0pt}{0pt}\eqref{floor-switch-ge}}{\iff}\quad
    \left\lfloor \dfrac{x\vphantom{l}}{n} \right\rfloor
    \le \left\lfloor \dfrac{x - 1}{n} \right\rfloor
\end{align*}
We can also leverage the result \eqref{floor-lt2ge-int}
and the definition of the floor function
$\left\lfloor \dfrac{x - 1}{n} \right\rfloor \le \dfrac{x - 1}{n}$ in:
\[
    -1 < 0
  \quad\iff\quad
    x - 1 < x
  \quad\stackfirst{n > 0}{\iff}\quad
    \dfrac{x - 1}{n} < \dfrac{x}{n}
  \quad\stackfirst{\text{(Floor)}}{\implies}\quad
    \left\lfloor \dfrac{x - 1}{n} \right\rfloor < \dfrac{x}{n}
  \quad\stackfirst{\eqref{floor-lt2ge-int}}{\iff}\quad
    \left\lfloor \dfrac{x - 1}{n} \right\rfloor
    \le \left\lfloor \dfrac{x\vphantom{l}}{n} \right\rfloor
\]
That is,
$\left\lfloor \dfrac{x - 1}{n} \right\rfloor =
 \left\lfloor \dfrac{x\vphantom{l}}{n} \right\rfloor$
when $x/n \notin \mathds{Z}$.
But in this case we also know that
$\left\lceil \dfrac{x\vphantom{l}}{n} \right\rceil -
 \left\lfloor \dfrac{x\vphantom{l}}{n} \right\rfloor = 1$.
Hence:
\[
  \left\lceil \dfrac{x\vphantom{l}}{n} \right\rceil
  = 1 + \left\lfloor \dfrac{x\vphantom{l}}{n} \right\rfloor
  = 1 + \left\lfloor \dfrac{x - 1}{n} \right\rfloor
\]
Which is the same result found for $x/n \in \mathds{Z}$.
Therefore, we conclude that:
\begin{equation}\label{ceil-x-over-n-as-floor}
  \left\lceil \dfrac{x\vphantom{l}}{n} \right\rceil
  = \left\lfloor \dfrac{x - 1}{n} \right\rfloor + 1
  \quad \forall x \in \mathds{Z}, \; \forall n \in \mathds{N}^*
\end{equation}
Such value is precisely the initial value equation
chosen for the algorithm.
