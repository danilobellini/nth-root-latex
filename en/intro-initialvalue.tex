\subsection*{Choosing the initial value}

The initial value $a_0$ could have been $x$ itself,
since $x \ge \lfloor \sqrt[n]{x} \rfloor$
for all positive integer $x$,
that is, it's a value that satisfies
the constraint defined by the algorithm.
Yet, the chosen value is typically less than $x$:
\[\tag{Initial value}
  a_0 = \left\lfloor \dfrac{x - 1}{n} \right\rfloor + 1
      = \left\lceil \dfrac{x\vphantom{l}}{n} \right\rceil
\]
For large values of $x$, this initial value will be close\footnote{
  As $\lceil x/n \rceil - 1 < x/n \le \lceil x/n \rceil$
  from the definition of the ceiling function,
  we have $0 \le \lceil x/n \rceil - x/n < 1$.
  Assuming $n$ is constant, the difference $\lceil x/n \rceil - x/n$
  becomes less significant as $x$ increases.
  \[
    \lim_{x\to\infty} \dfrac{\lceil x/n \rceil}{x/n}
    =
    \lim_{x\to\infty} \cancelto{0\;(\text{Limited}/\infty)}{
                        \dfrac{\lceil x/n \rceil - x/n}{x/n}
                      } +
                      \dfrac{x/n}{x/n}
    = 1
  \]
}
to $x/n$.
The justification for the equivalence
of the two expressions for the initial value
consists of deducing the equation \eqref{ceil-x-over-n-as-floor},
which is provided after identifying the properties
of the involved functions.
For everything else in this text,
the initial value is described only by means of the floor function.

Applying the floor function
never result in a value greater than the original one
upon which it's applied\footnote{
  The definition of the floor function is given later in the text,
  for now it suffices to know that $\lfloor \gamma \rfloor \le \gamma$.
},
allowing us to write:
\[
  \overbrace{\left\lfloor \dfrac{x - 1}{n} \right\rfloor}^{a_0 - 1}
  \le \dfrac{x - 1}{n}
  \quad\iff\quad
  a_0 \le \dfrac{x - 1}{n} + 1
\]

When is this choice of $a_0$
farther from the solution
than the adoption of $x$ itself as the initial value?
That is, which are the values of $x$ and $n$
that cause $a_0 > x$?
In order to answer these questions,
let's take $a_0 > x$ as a hypothesis,
so that $x < a_0 \le \dfrac{x - 1}{n} + 1$.
Thus:
\[
  x < \dfrac{x - 1}{n} + 1
  \quad\iff\quad
  x - 1 < \dfrac{x - 1}{n}
  \quad\stackfirst{n > 0}{\iff}\quad
  n(x - 1) < x - 1
  \quad\iff\quad
  (n - 1)(x - 1) < 0
\]
As $x$ and $n$ are positive integers,
such inequality is a contradiction,
allowing for us to say that $a_0 \le x$.
More generally,
we can group the inequalities of integer numbers:
\begin{equation}\label{a0-between-result-and-x}
  \lfloor \sqrt[n]{x} \rfloor
  \le
  \left\lfloor \dfrac{x - 1}{n} \right\rfloor + 1
  \le
  x
\end{equation}
The first part of the inequality, with the $n$-th root,
is demonstrated later, in \eqref{a0-not-below-result},
but the second part is already justified when $n \ge 1$ and $x \ge 1$.
