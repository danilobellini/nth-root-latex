\subsection*{Floor function}

Let $\gamma \in \mathds{R}$ and $k \in \mathds{Z}$.
One path for defining the floor function
is to set it as the $\lfloor \cdot \rfloor : \mathds{R} \to \mathds{Z}$
function satisfying:
\begin{equation}\tag{Floor}
  \lfloor \gamma \rfloor \le \gamma < \lfloor \gamma \rfloor + 1
\end{equation}
That is,
$\lfloor \gamma \rfloor$ is the greatest integer
less than or equal to $\gamma$.

By adding $k$ to the inequalities in the definition,
the result also satisfies the floor function definition:
\[
  \overbrace{
    \underbrace{\lfloor \gamma \rfloor + k}_{\in \mathds{Z}}
  }^{\lfloor \mu \rfloor}
  \le \overbrace{\gamma + k}^{\mu} <
  \overbrace{
    \underbrace{\lfloor \gamma \rfloor + k}_{\in \mathds{Z}} + 1
  }^{\lfloor \mu \rfloor + 1}
\]
In other words:
\begin{equation}\label{floor-int}
  \lfloor \gamma \rfloor + k = \lfloor \gamma + k \rfloor,
  \quad \forall \gamma \in \mathds{R}, \; \forall k \in \mathds{Z}
\end{equation}

It's always possible to say that
$\gamma < k \implies \lfloor \gamma \rfloor < k$,
since, from the definition given for the floor function,
$\lfloor \gamma \rfloor \le \gamma$.
On the other hand,
result \eqref{int+1} tells us that
when $\lfloor \gamma \rfloor < k$ for a given $k \in \mathds{Z}$,
we get $\lfloor \gamma \rfloor + 1 \le k$,
and we can join this result
with the second inequality in the definition of the floor function,
$\gamma < \lfloor \gamma \rfloor + 1$,
which allows for us to say that $\gamma < k$.
That is, $\lfloor \gamma \rfloor < k \implies \gamma < k$.
In short:
\begin{equation}\label{floor-switch-lt}
  \gamma < k \iff \lfloor \gamma \rfloor < k,
  \quad \forall \gamma \in \mathds{R}, \; \forall k \in \mathds{Z}
\end{equation}

A similar property can be found
for the complementary inequation.
As $\lfloor \gamma \rfloor \le \gamma$,
we have $k \le \lfloor \gamma \rfloor \implies k \le \gamma$.
On the other hand,
as $\gamma < \lfloor \gamma \rfloor + 1$,
if we assume $k \le \gamma$,
we get $k < \lfloor \gamma \rfloor + 1$,
and since it's an inequality of integers,
result \eqref{int+1} allows for us to say that
$k < \lfloor \gamma \rfloor + 1 \iff
 k + 1 \le \lfloor \gamma \rfloor + 1 \iff
 k \le \lfloor \gamma \rfloor$.
Thus:
\begin{equation}\label{floor-switch-ge}
  \gamma \ge k \iff \lfloor \gamma \rfloor \ge k,
  \quad \forall \gamma \in \mathds{R}, \; \forall k \in \mathds{Z}
\end{equation}

The last provided justification
also applies when we assume $k < \gamma$,
so that:
\begin{equation}\label{floor-lt2ge-int}
  \gamma > k
  \implies
  \lfloor \gamma \rfloor \ge k
  \quad \forall \gamma \in \mathds{R}, \; \forall k \in \mathds{Z}
\end{equation}

From applying the definition of the floor function
on $\lfloor \gamma \rfloor / n$, $n \in \mathds{N}^*$,
and multiplying the result by $n$, follows:
\[
      \left\lfloor \dfrac{\lfloor \gamma \rfloor}{n} \right\rfloor
    \le
      \dfrac{\left\lfloor \gamma \right\rfloor}{n}
    <
      \left\lfloor \dfrac{\lfloor \gamma \rfloor}{n} \right\rfloor
      + 1
  \quad\iff\quad
      n \left\lfloor \dfrac{\lfloor \gamma \rfloor}{n} \right\rfloor
    \le
      \left\lfloor \gamma \right\rfloor
    <
      n \left\lfloor \dfrac{\lfloor \gamma \rfloor}{n} \right\rfloor
      + n
  \quad\stackfirst{
    m =
    \left\lfloor
      {}^{\scriptstyle\lfloor \gamma \rfloor} / {}_{\scriptstyle n}
    \right\rfloor
  }{\iff}\quad
    n m \le \left\lfloor \gamma \right\rfloor < n m + n
\]

Result \eqref{floor-switch-lt}
applied to the inequality $\left\lfloor \gamma \right\rfloor < n m + n$
gives $\gamma < n m + n$,
and result \eqref{floor-switch-ge} tells us that
the inequality $n m \le \left\lfloor \gamma \right\rfloor$
can be written as $n m \le \gamma$.
Resuming:
\[
  n m \le \left\lfloor \gamma \right\rfloor < n m + n
  \quad\stackfirst{\eqref{floor-switch-lt}\text{ and }
                   \eqref{floor-switch-ge}}{\iff}\quad
  n m \le \gamma < n m + n
  \quad\stackfirst{n > 0}{\iff}\quad
  m \le \dfrac{\gamma}{n} < m + 1
\]
Which again consists in the definition of the floor function,
since $m$ is an integer.
That is, $m = \lfloor \gamma / n \rfloor$.
Finally:
\begin{equation}\label{floor-nested-ratio}
  \left\lfloor \dfrac{\lfloor \gamma \rfloor}{n} \right\rfloor
  =
  \left\lfloor \dfrac{\gamma\vphantom{l}}{n} \right\rfloor
  \quad \forall \gamma \in \mathds{R}, \; \forall n \in \mathds{N}^*
\end{equation}
